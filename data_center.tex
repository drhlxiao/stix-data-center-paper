\documentclass[referee]{preaa} % for a referee version
%\documentclass[onecolumn]{aa} % for a paper on 1 column  
%\documentclass[longauth]{aa} % for the long lists of affiliations 
%\documentclass[letter]{aa} % for the letters 
%\documentclass[bibyear]{aa} % if the references are not structured 
%                              according to the author-year natbib style


%\documentclass{aa}  
\usepackage{hyperref}
\usepackage{graphicx}
%%%%%%%%%%%%%%%%%%%%%%%%%%%%%%%%%%%%%%%%
\usepackage{txfonts}
%\usepackage{multirow}
%%%%%%%%%%%%%%%%%%%%%%%%%%%%%%%%%%%%%%%%
\usepackage{subcaption}
\usepackage{flushend}
  % results heading (mandatory)
\begin{document} 


   \title{The data center for the X-ray spectrometer/imager STIX onboard Solar Orbiter}

   \subtitle{}

   \author{Hualin Xiao
          \inst{1}
          \and 
          Shane Maloney 
          \inst{2}
          \and S\"am Krucker\inst{1}
          \and Ewan Dickson \inst{4}
          \and Paolo Massa \inst{6}
          \and Lastufka Erica \inst{1}
          \and Andrea Francesco Battaglia\inst{1,3}
          \and László Etesi \inst{1}
          \and Nicky Hochmuth \inst{1}
          \and Frederic Schuller \inst{7}
          \and Ryan Daniel \inst{1}
          \and Olivier Limousin \inst{5}
          \and Collier Hannah \inst{1,3}
          \and Alexander Warmuth \inst{7}
         }

   \institute{University of Applied Sciences and Arts Northwestern Switzerland (FHNW), 5200 Windisch, Switzerland \\
              \email{hualin.xiao@fhnw.ch}
         \and
          Astronomy and Astrophysics Section, School of Cosmic Physics, Dublin Institute of Advanced Studies, 31 Fitzwilliam Place Dublin 2, D02XF86, Ireland
          \and
             ETH Z\"urich, R\"amistrasse 101, 8092 Z\"urich, Switzerland
         \and University of Graz, Universitätspl. 3, 8010 Graz, Austria
         \and IRFU, CEA, Université Paris-Saclay and Université Paris Diderot, AIM, Sorbonne Paris Cité, CEA, CNRS, 91191 Gif-sur-Yvette,
         France
      \and Department of Physics \& Astronomy, Western Kentucky University, Bowling Green, KY 42101, USA
         \and Leibniz-Institut für Astrophysik Potsdam (AIP), An der Sternwarte 16, D-14482 Potsdam, Germany
             }

   \date{\today}

% \abstract{}{}{}{}{} 
% 5 {} token are mandatory
 
  \abstract
  % context heading (optional)
   { The Spectrometer/Telescope for Imaging X-rays (STIX) is the instrument onboard Solar Orbiter that measures spectra and takes images of solar flare X-rays in the energy range of 4 to 150 keV over a wide range of sizes.} %leave it empty if necessary  
  % {context.}
  % aims heading (mandatory)
   {During nominal operation, STIX continuously generates data. A constant data flow requires fully automated data processing pipelines to process and analyze the data and a data platform to manage, visualize and distribute STIX data generated by the pipelines.
   }
   {
   The STIX data center operated at FHNW consists of automated processing pipelines and a data platform.  The pipelines process STIX telemetry data, perform standard scientific analysis, and generate data products at different processing levels for delivery to ESA. The data platform provides web-based user interfaces and programmable application interfaces. }
   {
   The platform provides all STIX data products at different processing levels and offers users various web-based tools for searching, browsing, and downloading STIX data products and performing common analysis tasks. The data center is designed to operate fully automatically with minimal human intervention. The concept has proven successful and has been running continuously for over two years. The platform facilitates instrument operations and provides great support to STIX data users.}
 {}
\keywords{Solar flares -- Data platform --
                STIX data products --
                -- X-ray imaging -- 
                Data processing pipeline
               }
  \titlerunning{STIX data center}
  \authorrunning{Hualin Xiao and STIX team}
   \maketitle

%-------------------------------------------------------------------

\section{Introduction}
Solar Orbiter is a Sun-observing mission of the European Space Agency (ESA) that 
addresses the interaction between the Sun and the heliosphere. It was launched on Feb. 10, 2020, for a nominal mission duration of seven years and a planned 
extension of three years. It carries ten sets of instruments for comprehensive
remote sensing and in-situ measurements.
Solar Orbiter will perform detailed measurements of the Sun as close as 0.28 AU and, for the first time, will look at its uncharted polar regions \citep{SolarOrbiter2020}.  
The Spectrometer Telescope for Imaging X-rays (STIX) is one of the ten instruments onboard the Solar Orbiter mission.  
It measures X-rays from 4 to 150 keV and takes X-ray images with a few arcsec angular resolutions using an indirect imaging technique based on the Moiré effect. The STIX instrument performs measurements by means of 32 sub-collimators with grids and pixelated cadmium telluride detectors. The main science objective of STIX is to study the extremely hot solar plasma, and the high-energy electrons accelerated during solar flares. STIX provides information on the intensity, spectrum, timing, and location of accelerated electrons during solar flares. For more information on STIX instrumentation and its scientific capabilities, we refer to the instrumentation paper \citep{stix2020}.

During normal operations, STIX observes and acquires data continuously. Housekeeping (HK) and quicklooks (QL) are sent automatically at regular intervals, whereas other science products are requested from the ground. STIX compresses and structures the data into different types of telemetry packets onboard to reduce the amount of data to be transmitted.

The number of telemetry packets for larger bulk science data requests can reach hundreds of packets. To assist with the complexity of STIX data analysis, make the data more accessible to the solar physics community, and allow for regular data deliveries to ESA, automated data processing pipelines and a data platform have been developed at FHNW. The pipelines convert raw telemetry data into quick-looks and data products at different processing levels that can be used for scientific analysis. The platform stores all STIX data products and provides web-based tools for users to explore and analyze STIX data.

 This paper aims to describe the processing pipelines, the core algorithms of standard analysis, the STIX data products, and the tools the data platform provides for STIX data users.

\section{STIX telemetry data products}
\label{sec:raw-data}
For the most part, STIX data are organized by data content and type, like X-ray imaging data, spatially averaged X-ray data, housekeeping data, etc., and are associated with their specific telemetry packet types. While the format of each packet type is prescribed, the relative frequency of each packet type depends on solar activity, instrument mode, and configuration parameters. STIX generates many different types of raw telemetry packets.

However, from a scientific user perspective, only a few are relevant. They can be classified into three categories: housekeeping data, quick-look data, and bulk science data.
Housekeeping and quick-look data are directed to the low-latency data stored in the spacecraft's solid-state memory (SSMM) and sent to the ground with the highest priority.

Except for the raw pixel data product, all count- and trigger-based values in quick-look and bulk science data products are compressed with an integer compression algorithm (\citep{stix2020}).

\subsection{Housekeeping (HK) data}
 The housekeeping data monitor the instrument's status and performance and ensure it functions properly. STIX generates HK packets continuously while turned on. The housekeeping packets include details such as the instrument's temperature, voltages, currents, CPU usage, the status of the attenuator switches, trigger rates, file system information, and aspect system readouts. STIX typically generates a  housekeeping packet every 64 seconds, which results in a daily total of 143 KiB of raw telemetry data.
 
%Housekeeping data 
\subsection{Quick-look (QL) data}
The quick-look data are only generated when STIX is in NOMINAL mode, as they require the detectors to be powered and operating. There are four types of quick-look data:
\begin{itemize}
\item Quick-look light curves. Quick-look light curves contain time series of 4-second-integrated detector-summed counts in five energy bands, the corresponding detector-summed triggers, and the rate control regime state \citep{stix2020}, a measure to reduce detector sensitivity. Note that the counts of the two special detectors, the background monitor (BKG) and the coarse flare locator (CFL), are excluded. STIX does not correct any count data onboard for dead time, transmission, impacts of the presence of the attenuator, or rate control regime states.

\item Quick-look background light curves. STIX uses the background monitor, a detector with a special grid aperture, to monitor both the X-ray background and the intense unattenuated X-ray fluxes from large solar flares. It consists of an open front grid window and a rear grid window that is fully opaque except for six small openings. Quick-look background light curves contain counts and triggers of all BKG detector pixels, integrated over 8 seconds, in the same five energy bands as the quick-look light curves. 

\item Quick-look variance data are the onboard computed variance of 40 successive detector-summed count rates based on 0.1-second integration.

\item Quick-look spectra. Quick-look spectra are snapshots of the energy spectra with an accumulation time of 32 seconds, where each spectrum is integrated over all pixels in the same detector. STIX takes a snapshot of the energy spectra during nominal operation every 1024 seconds.

\item Calibration spectra. Calibration spectra are high-resolution raw energy spectra with full pixel resolution, measuring photons emitted by $^{133}$Ba radioactive sources. STIX generates a calibration spectrum for each pixel every 24 hours during normal operations. Calibration spectra are used to determine the energy conversion factor of each pixel by using the photon peaks in the calibration spectra on the ground.  
\end{itemize}

\subsection{Bulk science data (BSD)}
Bulk science data products are different combinations of pixel and detector sums of the raw pixel data stored in the STIX onboard archive memory. BSD products are only generated and down-linked in response to a data request from the ground. They do not present a continuous record of solar activity due to limited bandwidth and onboard storage, forcing the data to be overwritten after a few weeks. STIX can generate six different types of bulk science data products: 
\begin{itemize}
 \item Raw pixel data, which are the least processed data product. It provides time-binned pixel counts in the highest available time and energy resolution, generally with a dynamic accumulation time between 0.5s and 20s, depending on solar activity. The raw pixel data product values are not integer-compressed, require a large amount of telemetry, and are primarily used for testing and verification.
 
\item Compressed pixel data, which are based on the raw pixel data product and combine the highly-resolved pixel counts into larger user-defined time-energy bins. In addition to the re-binning, the pixel data are also integer-compressed onboard before being sent to the ground. This is the main requested bulk science data product, as it is well-suited for scientific analysis. 

\item Summed and compressed pixel data, which are based on the raw pixel data product and combine the raw pixel counts into larger user-defined time-energy bins and sum over multiple pixels. The pixel sums can be configured by telecommand. By default, the counts of the bottom and top row pixels are individually summed, i.e., A top with A bottom, B top with B bottom, etc. The summed pixel data values are integer-compressed. The STIX instrument team only requests this data product when the available telemetry bandwidth is low.

\item Visibility data, which are further reduced by calculating complex visibility components from the four summed pixel counts from the previous data product. Visibility components can be used for imaging but require accurate onboard energy calibration. This product is not requested, except for testing, and is only foreseen during extremely low telemetry rates.

\item Spectrogram data, which are summed over selected pixels, detectors, and time-energy bins. Along with pixel data, spectrograms are the most requested types of bulk science data products. Spectrograms can be used for the spectral analysis of flares.

\item High-time-resolution aspect data.  STIX stores high-time-resolution readouts from the four photodiodes in the aspect system in the onboard archive memory. They can be requested to derive accurate STIX pointing centers for periods when the spacecraft attitude changes rapidly.
\end{itemize}

Bulk science data are transmitted to the ground in raw binary packets format and considered level-0 at the STIX data center.  

\section{Data reception}
During nominal science operations, low-latency data are transmitted at every ground station pass, regardless of orbital geometry, whereas bulk science data are only downlinked when bandwidth permits. Every instrument team is responsible for keeping the transmitted data volume within pre-allocated boundaries.

ESA's Ground Operation System (EGOS) processes telemetry data received by ground stations at the Solar Orbiter mission control center. Then the telemetry data are distributed to the instrument teams regularly via the EGOS Data Dissemination System (EDDS) \citep{EDDS}. 

Low-latency telemetry data generally arrive at the STIX data center within a few days, depending on the next ground station pass. The bulk science data, however, may be delayed by several weeks to a few months after being generated onboard due to possibly low bandwidth and bandwidth allocations.

In addition to STIX instrument telemetry data, the STIX data center receives auxiliary data \citep{spice1996,spice2018} from the science operations center, which contain information on spacecraft ephemeris, attitude, and calibration factors required for the conversion of onboard timestamps to UTCs. 

\section{Data processing pipelines}
\subsection{Data processing pipeline overview}

\begin{figure*}
    \centering
    \caption{Telemetry processing pipelines at STIX data center.}
    \label{fig:main_pipelines}
\end{figure*}
STIX telemetry data are processed by the data processing pipeline  
as shown in Fig. \ref{fig:main_pipelines} immediately after reception at the STIX data center.  It starts with level-1 processing of raw packets, which includes several steps such as  packet parsing,  integer decompressing, and timestamp conversion. 
Packets at the L1 level stored in a tree-like structure  are written to a NoSQL database. Then they are selected and processed in four different paths. 
In the first path, the housekeeping, quick look and science data are successively selected and used to create level-1 FITS files. 
The second path selects calibration data for the determination
 of energy calibration factors.   In the third path, quick-look packets are selected  for the identification of solar  flares; 
The fourth path performs standard analysis of science data for flares.

\subsection{Raw packet parsing}
Raw telemetry data at the STIX data center are in the form of binary packets. 
Each packet contains a fixed-length header and a list of parameters that vary with the type of packet.  The parsing of parameters is based on the information in 
the mission interface database (MIB), which contains the name and the 
length of each parameter for each type of packet. 
The packets after parsing contain raw values of parameters, 
which need to be converted to physical values. 
Raw values of spacecraft-local times are converted to UTC times by using 
the latest version of SPICE kernels \citep{spice1996,spice2018};
Raw values of  housekeeping parameters are converted to physical values using 
the  ground-calibrated conversion factors stored in the MIB. 
For compressed counts in science data, they are decompressed using a look-up table. 
After the above processing steps, packets are organized in a tree-like structure. 
They are considered level-1 packets and written to a collection in the NoSQL database. 
The NoSQL database is schemaless, in which the data format of each record can be different, and pre-define the data format is not required, so it is very suitable for storing the tree-like structure level-1 data packets.  
The use of the NoSQL database provides great convenience for finding, sorting, merging packets, and also checking data integrity. 

In addition to level-1 packets,  other metadata such as filenames, version of SPICE kernel data, and version of the MIB are written to another collection in the NoSQL database, which allows fast query of the associated data of the raw telemetry data file.

\subsubsection{FITS products creation}
The Flexible Image Transport System (FITS)  is a portable file standard widely used in the astronomy community to store, transmit and manipulate scientific images, tables, and associated data \citep{fits}.
Therefore, the FITS format is adopted by the STIX data center to store the standard data products. 
After the parsing of each new raw telemetry file, housekeeping, quick look, and science packets are sequentially selected from the NoSQL database and merged after passing checks for data integrity and consistency.  The merged data as well as the associated metadata and auxiliary data are written to FITS files.  In the meantime, their metadata are written to a collection in NoSQL, which allows for fast queries on the products. The FITS files, created by level-1 packets immediately upon the parsing process, are defined as level-1A (pre-released level-1) products.  They are used by some subsequent data processing pipelines.  
FITS files are recreated after a few days to weeks after all resources are validated.
They are regarded as the formal level-1 products at the STIX data center. 
In most cases, the FITS files at level-1A and level-1 are almost identical, except that the level-1A FITS files may use predicted ephemeris data. As such, FITS files at level-1 FITS files are recommended for STIX data users when they are available. 

\subsection{Energy calibration}
%STIX performs energy calibration onboard by using an energy look-up table, 
%which has to be
\begin{figure}
 \centering
  \includegraphics[width=0.8\linewidth]{figures/cal-fit.pdf}
  \caption{An example of STIX in-flight calibration spectrum.
  The most prominent peaks, from left to right, are photo-peaks of 31, 35, and 81 keV
  photons. The first two peaks are fitted by the double-Gaussian function, and the high energy peak by  
  the crystal-ball function. }
    \label{fig:cal-fit}
\end{figure}
STIX converts ADC channels to "science energy channels" using an energy lookup table (ELUT) 
onboard by the FPGA, which defines the ADC channel edges  for each science channel for each pixel. 
A new ELUT is uploaded to STIX once a significant gain change is observed during the data analysis on the ground.  An ELUT can be constructed using energy conversion factors determined from calibration runs. 
STIX continuously accumulates an  energy spectrum (in ADC units) for each pixel separately for events from the onboard Ba$^{133}$ sources and formats a  spectrum typically every 24 hours. 
The right panel of Fig.~\ref{fig:cal-fit} shows an example of the calibration spectrum with an accumulation time of 24 hours.  The three most prominent peaks are produced by photons of 31 keV, 35 keV, and 81 keV photons from the calibration sources. To determine the positions of the photo-peaks, the first two peaks are fitted with the double-Gaussian function, and the third peak with the crystal-ball function \citep{crsystallball},
which consists of a Gaussian core portion 
and a power-law low-end tail, below a certain threshold.
Then a linear line is fitted to the positions and keV energies.  That gives the gain (i.e., the ADC to energy conversion factor) and baseline.
The ECC method (see \citet{ecc,ecc2}) is another method that the STIX team often uses to determine the calibration factors.  We found that the results of the two methods are consistent at 1$\sigma$.

The above steps are performed for each calibration spectrum once the data are available at the STIX data center. 
The calibration factors are written to a collection in the NoSQL database and used for further correction of energy bins in offline data analysis.   
Once significant changes in the calibration factors are observed, the STIX operations team creates a new ELUT and uploads it to STIX.

\subsection{Solar flare identification}
STIX identifies solar flares onboard based on detector count rates, and the results are included in the QL data \citep{stix2020}.  However,  the data only provide limited information on flares due to the constraints of the telemetry budget and onboard computing resources.  Moreover, micro-flares are not reported due to the relatively high trigger threshold.
It is necessary to maintain a flare list on the ground for the needs of requesting science data and also for data users to find events of interest. 

\begin{figure}
  \centering
  \includegraphics[width=0.8\linewidth]{figures/flaredet.pdf}
  \caption{STIX 4 -- 10 keV QL light curve recorded from 2022-08-10T21:00:00Z to 2022-08-10T18:00:00Z and  identified flares. The orange curve is the smoothed light curve using a moving average filter with a time window of 1 minute. 
  The identified peaks are marked with plus signs, and the colored ranges show their time ranges.
  }
  \label{fig:flare-det}
\end{figure}
Using QL light curves, solar flares can be identified in greater detail on the ground. 
The ground identification procedure includes the following steps:
\begin{itemize}
  \item Light curve smoothing: The selected light curve is filtered using an unweighted
  moving average filter with a time window of 1 min. This can smooth out statistical fluctuations and electric surge spikes;
  \item Envelope subtraction:  A flare may last hours, and there may be short-duration pulses lying on the envelope (the main pulse) in the light curve.  To facilitate the identification of those short-duration pulses, the envelope is subtracted from the smoothed light curve, which is estimated using the SNIP algorithm \citep{snip}. 
  \item Identification of flare peaks: We consider a flare is detected 
 if the peak count rate after envelope subtraction lies beyond two standard deviations 
 of the mean count rate during quiet Sun periods. 
  The flare start and stop times are given at the times crossing the threshold.  
  \item Merging of flare peaks: Two flares are considered as one single flare if their peak times differ by less than 5 minutes. This can reduce the number of flares reported and facilitate subsequent data analysis.  
\end{itemize}

As an example, Fig.~\ref{fig:flare-det} shows STIX QL light curve in the energy range 4 -- 10 keV, recorded from 2022-08-10T10:00:00Z to 2022-08-10T18:00:00Z.  The orange curve is the smoothed light curve.  The identified peaks are marked with the plus signs, and the colored ranges show the time ranges.

The above steps are repeated for QL light curves of the other four higher energies, which provide information on the upper limit of the X-ray energy of the flare.
The time ranges, peak count rate, and total counts as well as the corresponding ephemeris data of the identified flares are  stored in a collection called flare list in the NoSQL database.


\subsection{Solar flare standard analysis pipeline}
\subsubsection{Estimation of solar flare GOES class }
\begin{figure}
  \centering
  \includegraphics[width=0.8\linewidth]{figures/goes_stix_flux_paper.pdf}
  \caption{Scatter plot of GOES low channel peak flux with respect to the equivalent  peak count rate  at 1 AU  in the 4 -- 10 keV range for 717 solar flares observed by both GOES and STIX duration of the commissioning phase.   The solid line is a linear fit to the log-log graph. }
\label{fig:goes-stix}
\end{figure}
Solar Orbiter is far from Earth most of the time. 
Therefore, a considerable number of flares observed by STIX 
are not observed by GOES satellites (and vice versa). 
In order to estimate the GOES classes of such flares, 
we selected 717 solar flares observed by both GOES and STIX in 2021. 
Fig.~\ref{fig:goes-stix} shows the scatter plot of the peak fluxes
measured by GOES satellites with respect to the STIX background-subtracted count rates at the peaks, 
 in the energy range of 4 to 10 keV. 
STIX count rates have been corrected for 
the different distances between the Solar Orbiter and the Sun using $X^{'}=x r^2$,
where $X$ is the count rate after background subtraction
 and $r$ is the distance between the Sun and Solar Orbiter in units of AU.  There is a clear correlation between them, as can be seen in the figure.  The widespread at low fluxes can be explained by the difference in 
the energy response of the two instruments and the variation in flare temperatures. 
The correlation can be fitted with a linear fit in the log-log scale. 
From the fit, we get the GOES flux estimation formula as follows: 
$f = 10^{0.622 -7.376 \log_{10} (X^{'})}$ (in units of W/m$^2$), where $X^{'}$ is the STIX peak count rate corrected for the distance variations between the Sun and Solar Orbiter. It is currently used to estimate the GOES classes of flares that are not directly observed by GOES satellites.  The estimated GOES fluxes are stored in the flare list collection in the database. 

\subsubsection{Estimation of coarse flare locations using CFL data}
STIX estimates coarse flare locations onboard by 
maximizing the correlation between observed CFL pixel counts with expected counts using a lookup table \cite{stix2020}. 
With the requested science data, coarse flare locations can be reconstructed more accurately, as it allows for more sophisticated algorithms, greater flexibility in selecting time and energy range to be integrated, and more careful background subtraction.

When the pixel data of a flare are available at the STIX data center, its flare location is estimated. The steps are as follows: 
1) Integrate counts around the peak for each CFL pixel, 
2) Subtract the background using a background file, and 3) estimate the illuminated area on each CFL pixel.  The estimation is based on two assumptions:  the illuminated area of a pixel is proportional to its relative count rate, and the total illuminated area of the imaging detectors is independent of the source location.
4) Estimated the flare location by minimizing the weighted sum of squared deviations (i.e., weighted chi-squares) between the calculated illuminated areas and expectations simulated for potential flare locations in a 400 $\times$ 400 grid.
\begin{figure*}
  \centering
  \includegraphics[width=0.95\linewidth]{figures/cflMay07.pdf}
  \caption{
   Left: Calculated areas of illuminated regions of 12 CFL pixels and the best fit simulated pattern for the flare location at (550, 278) arcsec. Pixels 0 to 3 are the top big pixels, Pixels 4 to 7 are the bottom pixels, and 8 to 11 are the small pixels as shown in the right panel.
   Middle: Best-fit flare centroid location (marked by x) and its 1$\sigma$, 2$\sigma$, and 3$\sigma$ confidence contours. Right: Projection of CFL sub-collimator (the gray shaded regions) on the detector simulated for the best-fit flare location. }
  \label{fig:cfl}
\end{figure*}
As an example, the left panel of Fig.~\ref{fig:cfl} shows the calculated and best-fit illuminated areas of the CFL pixels for the flare observed at 2021-05-07T19:00:00 (GOES class M3.9);  the middle panel shows the best-fit flare centroid location, as well as its $1\sigma$, $2\sigma$, and $3\sigma$ contours. 
The simulated CFL shadow pattern is shown in the right panel. 
Flare location sources are stored in the flare list in the database. 

\subsubsection{Imaging and spectroscopy pipeline}
\begin{figure*}
  \centering
  \includegraphics[width=0.95\linewidth]{figures/imaging_pipeline.pdf}
  \caption{ 
   STIX Quick-look light curves and reconstructed images of the solar flare observed at 2022-03-08T08:55:17Z, 
   created by the image reconstruction pipeline.  A 1-min integration time around the peak was selected and Back Projection, CLEAN, MEM\_GE and VIS\_FWDFIT were used for reconstructing the images. 
The full-disk Back Projection image shown in the top-row middle panel has been used for identifying the source location. The other reconstructions were performed considering a Field of View (FOV) of dimension 256 arcsec $\times$ 256 arcsec around the location of the flare. Note that a Gaussian elliptical shape has been selected for reconstructing the flaring source by means of VIS\_FWDFIT (bottom-right panel). }
  \label{fig:imaging}
\end{figure*}
STIX detects thousands of solar flares each year. However, only a small part of them is analyzed in detail by solar physicists. 
To help in finding flares of interest, we developed an imaging and spectroscopy pipeline that automatically reconstructs images and performs spectral analysis for each ground identified flare after the reception of its pixel data. 

 For each flare, the pipeline first selects and integrates counts for 60 seconds around the peak of the flare for each pixel and then subtracts the background from the integrated counts using the pixel data acquired during a quiet solar period. 
 Subsequently, the transmission and dead time corrections are performed on the counts after background subtraction. The corrected counts are further converted into the visibilities of two energy bands of 4 -- 10 keV (thermal energy) and 16 -- 28 keV (non-thermal energy). 
Then the visibilities are used for image reconstruction using four different algorithms: back-projection \citep{paolo2022}, CLEAN \cite{clean}, MEM\textunderscore GE \citep{mem, memge} and VIS\textunderscore FWDFIT \citep{visfwd}.
The reconstructed images are finally corrected  for STIX off-pointing and rotations. 
As an example,  the first panel of Fig. ~\ref{fig:imaging} shows the light curves and time range selected for a flare that occurred at 2022-03-08T08:55:17Z. The rest of the panels show the images reconstructed with the algorithms. 
In addition to the image reconstruction, the counts after the transmission and dead time correction are used for spectral analysis.  Fig.~\ref{fig:ospex} shows an example of the spectral fitting results. The spectrum is fitted with a thermal component and a non-thermal component \citep{andrea2021}. 

The results of the pipeline are saved to files in both FITS and PNG formats. At the same time, the file indexing information, parameter values from spectral analysis, and auxiliary data are written to a collection in the database. 


\begin{figure}[h]
  \centering
  \includegraphics[width=0.9\linewidth]{figures/ospex.pdf}
  \caption{An example of spectral fitting results. 
    The imaging and spectroscopy pipeline performs
    spectral analysis for all flares in the flare list. }
  \label{fig:ospex}
\end{figure}
\section{Science data request strategy}
As mentioned earlier, the high-resolution pixel data are only downlinked in response to requests from the
ground. They are stored in the onboard archive memory for about four to six months 
before being overwritten by new data. They are processed and downlinked 
after receiving data request telecommands from the ground. 
A data request telecommand contains information about the selected data and values of parameters  required to process the data onboard, including the data compression level, 
time range, minimal time bin, energy bin width, 
and  masks indicating detectors and pixels to be selected. 
STIX data detect thousands of flares per year; therefore, 
selecting data is a tedious task, as it must consider many factors
such as count rates, time binning of data, statistics of selected data, 
and also the telemetry budget. 
The data selection strategy has been continuously optimized over the past two years. The current strategy is as follows: 
\begin{itemize}
  \item  
 Compression level-1 pixel data are requested for each of the ground-identified  flares with a total number of signal counts in the QL light curves greater than 10000, which is approximately the minimum counts to reconstruct an X-ray image. 
The requested energy range is chosen to be the range in which obvious signal counts are seen,  
whereas, the requested time resolution is adjusted based on the amount of data allocated. 
If the peak count rate is above 125 counts/sec (approximately 
equivalent to the count rate observed for a B3 flare at 1 au),  pixel counts with high time resolution are requested.  Otherwise, pixel counts are integrated over the whole flaring time 
to reduce the telemetry data volume. 
 \item  Spectrograms with the highest time and energy solution (compression level 4) are requested for all periods when STIX is in observation mode. 
 \item Time-integrated pixel data for background subtraction:
Time-integrated pixel data with durations of one or two detector temperature cycles (each cycle lasts about 40 minutes), which are acquired during quiet-Sun periods, are requested. 
The data are used for background subtraction when performing spectroscopy and imaging. 
\item High time-resolution aspect data are requested for periods when the spacecraft's attitude changes drastically.
Such periods can be known from the SPICE kernels or the aspect system readouts in housekeeping data. 
\end{itemize}
The selection of science data of the above types is done automatically using a program. The information of the selected data is written into a database collection.  In addition, the STIX operations team also selects data for special 
needs.  After being checked and adjusted by the STIX operations team, groups of new data requests that meet the operations requirements are selected from the database and then compiled into instrument operation requests (IORs). 
IORs are used to create final telecommands at the mission operations center (MOC). An IOR is typically executed after two to three weeks of submission to ESA. 

\section{STIX data platform user interfaces}
\subsection{Interactive web pages}
\begin{figure}[ht]
  \centering
  \includegraphics[width=0.9\linewidth]{figures/interfaces.pdf}
  \caption{ 
    Data flow at STIX data center.
  }
  \label{fig:interfaces}
\end{figure}

\begin{figure*}[ht]
  \centering
  \includegraphics[width=0.7\linewidth]{figures/data-browser.pdf}
  \caption{ 
    Interactive web-based STIX Quick-look data browser. 
    In addition to STIX Quick-looks, it can also display quick-looks of simultaneous measurements 
    performed by other instruments, such as GOES X-ray fluxes, SDO/AIA images, and Solar Orbiter EUI. 
    The browser is available at the link \url{https://datacenter.stix.i4ds.net/view/ql/lightcurves}.}
  \label{fig:qlbrowser}
\end{figure*}
The data center provides various HTTP interfaces (APIs) that allow access to STIX data products
and the NoSQL database via HTTP requests. 
We have built dozens of web applications to manage and browser STIX data based on the APIs. 
Web techniques are chosen because they offer many advantages, such as clear cross-platform 
usability, broad access through browsers, rapid development, and easy maintenance.

As an example, Fig.~\ref{fig:qlbrowser} shows a screenshot of the STIX Quick-look data browsing 
tool. It allows users to browse available Quick-look data interactively. It 
interacts with the server through APIs. 
After receiving a request from the client side,
the server retrieves QL counts from the NoSQL database for the user-specified time range. 
After excluding duplicates and merging,  the server sends QL counts and metadata in JSON format back to the client. The data are then used to create interactive light curve plots using JavaScript on the client side. 
The interactive plot uses state-of-the-art web technologies that enable users to perform a range of operations, such as rebinning the integration time, correcting the light travel time between the spacecraft and the Earth, and exporting data to a local file. In addition to STIX quick-looks, quick-looks from other solar-observing instruments can also be displayed on the same page after users' activation, 
making it easier to find events of interest for joint analysis.

Based on similar concepts, tens of web tools have been developed to browse other STIX data products.
The other four most commonly used tools are listed below: 
\begin{itemize}
  \item  Science data manager and browser.  It provides users with tools not only for visualizing science data, but also for searching, downloading, and analyzing science data. The interactive analysis tools allow users to interactively select data of interest for common analysis tasks, such as background subtraction and energy rebinning,  without installing additional software. 
  The algorithms are implemented client-side using JavaScript. 
 In addition, users can use the tools to submit imaging and spectroscopy tasks to the server and view the results on the same page. This reduces the barriers to exploring STIX data for new users and provides convenience for experienced users.

  \item  The preview images and spectroscopy product viewer is a web-based tool for managing and viewing the imaging and spectroscopy results. The viewer also provides tools for plotting the time evolution of emission measures and temperatures,  creating animations of X-ray images for the selected runs, generating IDL or Python templates in order to reproduce the same results on  local machines, and so on. 
  \item The auxiliary data viewer is a tool that allows the user to view auxiliary data, such as the locations of the spacecraft, its velocity, and its attitude. The viewer uses data derived from the SPICE kernel or from aspect system readouts stored on the server side.  The viewer also provides tools to calculate the looking angle of flares and the coordinates of solar limbs within STIX field of views.
  \item  With the housekeeping browser, users can view time-series plots of all STIX housekeeping parameters, including the temperature, voltage, operation mode, memory status, etc. It provides great convenience for the instrument operations team to monitor the instrument status. 
\item 
The STIX data access page offers users a variety of tools to search and download STIX FITS products. 
It also provides links to tools for previewing the products. As soon as they are generated at the STIX data center, STIX data products are immediately available for access on the page.
\end{itemize}




\subsection{STIX data center interface:  stixdcpy}
{\it stixdcpy} is a python package that facilitates accessing and analyzing STIX data. 
With {\it stixdcpy}, users can easily query and download the data products available at the STIX data center.
Similarly to the web tools, {\it stixdcpy} also provides tools to perform some common analysis of STIX data, such as dead time correction, transmission correction, data clipping, and merging. 
{\it stixdcpy} is still under active development. 
As a result, its features and 
capabilities may change over time.  The source code of {\it stixdcpy} is hosted on the GitHub repository at \url{https://github.com/i4Ds/stixdcpy}.
\section{Summary}
\label{sec:summary}
STIX is one of ten instruments onboard the Solar Orbiter, 
which was launched into space on February 10, 2020.
 STIX measures  
intensity and spectrum of hard X-rays emitted during solar flares
in the energy range of 4 -- 150 keV.  
 During nominal operations, STIX continuously generates telemetry data. 
 To process and archive the data as well as to support the operation of 
 instruments and scientific activities using STIX data, 
 dedicated data processing pipelines and a data platform have been 
 developed for STIX.
 The pipelines generate telemetry at different levels and perform common scientific analyses. 
 The platform provides 
 all STIX data products of different levels and also provides users 
 with various web-based tools for searching and browsing STIX data products. 
 It also provides web-based tools to perform common analysis tasks with STIX data. 
  The data center is designed to work in a 
 fully automatic mode with minimal human intervention. The concept has proven successful 
 and has been running continuously for more than two years.
The platform not only facilitates the operations of the instrument, but also provides great support to STIX data users.



\bibliographystyle{aa}
\bibliography{citations}

\end{document}
