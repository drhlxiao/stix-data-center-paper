%
\documentclass[referee]{aa} % for a referee version
%\documentclass[onecolumn]{aa} % for a paper on 1 column  
%\documentclass[longauth]{aa} % for the long lists of affiliations 
%\documentclass[letter]{aa} % for the letters 
%\documentclass[bibyear]{aa} % if the references are not structured 
%                              according to the author-year natbib style


%\documentclass{aa}  

\usepackage{graphicx}
%%%%%%%%%%%%%%%%%%%%%%%%%%%%%%%%%%%%%%%%
\usepackage{txfonts}
\usepackage{multirow}
%%%%%%%%%%%%%%%%%%%%%%%%%%%%%%%%%%%%%%%%
\usepackage{subcaption}

  % results heading (mandatory)
\begin{document} 


   \title{Data processing pipelines and the data center for the X-ray spectrometer/imager STIX}

   \subtitle{}

   \author{Hualin Xiao
          \inst{1}
          \and 
          Shane Maloney 
          \inst{2}
          \and 
          Ewan Dickson \inst{4}
          \and 
          S\"am Krucker\inst{1}
          \and Andrea Francesco Battaglia\inst{3}
            \and László Etesi \inst{1}
          \and Ryan Daniel \inst{1}
          \and Lastufka Erica \inst{1}
          \and Olivier Limousin \inst{5}
          \and Nicky Hochmuth \inst{1}
          \and Other STIX team members
         }

   \institute{University of Applied Sciences and Arts Northwestern Switzerland (FHNW), 5200 Windisch, Switzerland \\
              \email{hualin.xiao@fhnw.ch}
         \and
          Astrophysics Research Group, School of Physics, Trinity College Dublin, Dublin 2, Ireland
          \and
             ETH Z\"urich, R\"amistrasse 101, 8092 Z\"urich, Switzerland
         \and University of Graz, Universitätspl. 3, 8010 Graz, Austria
         \and IRFU, CEA, Université Paris-Saclay, and Université Paris Diderot, AIM, Sorbonne Paris Cité, CEA, CNRS, 91191 Gif-sur-Yvette,
         France
             }

   \date{}

% \abstract{}{}{}{}{} 
% 5 {} token are mandatory
 
  \abstract
  % context heading (optional)
   {} %leave it empty if necessary  
  % {context.}
  % aims heading (mandatory)
   { The Spectrometer/Telescope for Imaging X-rays (STIX) instrument onboard the Solar Orbiter mission launched on February 10, 2020 promises advances in the study of solar flares of various sizes. It is capable of measuring X-ray spectra from 4 to 150 keV with 1 keV resolution binned into 32 energy bins before downlinking. STIX data center is an infrastructure established at FHNW in order to process and archive STIX telemetry data, and to support the operations of the instrument. The automated data processing pipelines turn raw telemetry data into processed information and data products. Processed information and data products are achived at the data center.  STIX data center provides the solar physics community various tools to visualize STIX data products.
   }

   {Results.}
  % conclusions heading (optional), leave it empty if necessary 
   {}

   \keywords{Solar flares --Data Center --
                STIX data products --
                Data processing pipeline
               }

   \maketitle
%
%-------------------------------------------------------------------

\section{Introduction}
Solar Orbiter is a Sun-observing mission of the European Space  Agency that 
addresses the interaction between the Sun and the heliosphere.
It was launched on Feb. 10, 2020 for a nominal mission duration of seven years and a planned 
extension of
three years. It carries ten sets of instruments for comprehensive
remote-sensing and in-situ measurements. 
Solar Orbiter  will perform detailed measurements of the Sun as close as 0.28 AU and for the first time look at its uncharted polar regions (\cite{SolarOrbiter2020}).  
Its goal is to  address the center question of heliophysics  "How does the Sun create and control the heliosphere?".  It is designed to identify the origins and causes of the solar wind, the heliospheric magnetic field, the solar energetic particles, the transient interplanetary disturbances, and the Sun's magnetic field.
This consists of the study of energetic solar phenomena like flares,  solar transients,  the solar wind accelerating mechanisms, and the solar dynamo principle.  


The Spectrometer Telescope for Imaging X-rays (STIX) is one of the instruments onboard Solar Orbiter.  
It measures X-rays from 4 to 150 keV and takes X-ray images with a few arcsec angular resolution by using an indirect imaging technique based on the Moiré effect. 
STIX instrument consists of 32 collimators with
grids and pixelated Cadmium telluride detector units called Caliste-SO.
The main science objective of STIX is to study the extremely hot solar plasma and the high-energy electrons accelerated during solar flares. STIX will address the key science goals of the Solar Orbiter mission by providing information on intensity, 
spectrum, timing, and location of accelerated electrons near the Sun.
For details of the STIX instrumentation, we refer to the instrument paper (\cite{StixInstrument}).


During nominal science operations, STIX continuously acquires science data. 
They are processed and compressed into telemetry packets by
before they are transmitted to the ground.
Being aware of the complexity of STIX data analysis and 
the need of bringing the data to the solar physics community, a data center has been
developed at FHNW. The data center (SDC) receives, analyzes, archives and distributes STIX data,
 and supports STIX in-flight operations.
It turns raw telemetry data into processed information and produces data products that can be used for scientific analysis.
It also provides various data visualization tools to the solar physics community.
The purpose of this is to describe the standard processing pipelines, 
algorithms of automated data processing, data products and tools provided by the STIX data center.
%We will describe here STIX data types,  the flow of data from to the users, the data processing pipelines, the %database, the data products and the tools provided for the solar physics comnunity.
%--------------------------------------------------------------------

This paper is organized as follows: Section \ref{sec:raw-data} briefly introduces STIX raw telemetry types, Data flow and telemetry
We end with a summary in Section \ref{sec:summary}.
\section{STIX raw telemetry data}
\label{sec:raw-data}

STIX continuously observes high-energy events on the Sun at 4 -- 150keV. 
Photons emitted by solar flares are detected with 32 subcollimators 
(a 12-pixel-detector behind a front and rear grid). While passing through the front and rear grids, 
the flares generate a modulation pattern over the 12 pixels of each detector. 
The pattern can then be used to reconstruct images and to do spectroscopy. 
Other data products include lightcurves, flare information, spectra, and background information.
The nominal telemetry budget of STIX is 50 bps.
STIX is far from earth, not all data can be downloaded from the spacecraft. We have low latency data.
For bulk science data have to be requested.
STIX continuously collects energy deposition information from 32 detector units, the aspect system,
and engineering sensors in the nominal observation mode.
The collected data are first processed by the FPGA and the onboard flight software
After the prompt processing, low latency telemetry data are directed to the
storage in the spacecraft whereas high time resolution pixel data are written to STIX onboard archive memory for
later processings.
STIX transmits data to the spacecraft in the form of binary packets.
STIX raw telemetry data can be classified into four
categories: housekeeping data, diagnostic data, quick-look data and science data.

In the next sections, we briefly introduce the main raw telemetry data types.
For details about STIX raw telemetry data, we refer to STIX instrument paper (\cite{StixInstrument}).

\subsection{Housekeeping data}
 %\item  Housekeeping data.
 %STIX generates two different houskeeping data packets.
STIX housekeeping data (HK) contain engineering data that measure temperatures, voltages, currents, the status of switches,
averaged signal readouts from the four photodiodes in the aspect system, detector trigger counts, and the flags indicating
status bits of the onboard software and the archive memory.
The data is used to monitor instrument status, and the instrument pointing.
%All the parameters are reported in the same type packet at a fixed 64 sec cadence.
HK data are generated as long as STIX is powered on.
During nominal operations, a housekeeping packet is generated every 64 seconds, which results in a daily raw telemetry of  143 kiB.
All STIX sensors and the IDPU will produce housekeeping data that will be received on the ground with the highest priority
for monitoring instrument status and health.

\subsection{Quick-look data}
STIX has only one operational mode in which Low Latency data is produced, and that is  NOMINAL mode, which is the regular science mode.
STIX generates four different types of quick-look (QL) data:
\begin{itemize}
%of all pixels in 30 detectors (excluding background detector and the coarse flare locator)
%accumulated every 4 sec time resolution in five broad energy ranges (two thermal, two nonthermal
%and one intermediate). Quick-look light curves are generated when STIX is in nominal observation mode.
\item STIX light curves are a time series of detector counts in different energy bands (units will be counts/second/cm$^2$). There are five configurable energy bands (for example: 4-7keV, 7-11keV, 11-16keV, 16-40keV, and 40-150keV), with a default integration time of 4 seconds. 
A lightcurve data point corresponds to the total counts integrated over all selected detectors and pixels for the given energy band. 
The data are temperature corrected on-board, and undergo minor corrections for detector and pixels,
and nominal grid transmission (25 percent) on the ground. As lower energies are highly impacted by the presence of the attenuator, they will also need to be corrected. The data will not be corrected for background, live time or detector
% efficiency. The FITS file is in accordance with [LLFITSICD] but will also comply with  in order to be usable with existing high-energy solar data file routines.
%The light curve data are structured as an array of five count values COUNTS (for the five energy bands) 
%per time, where time is a relative time $RELATIVE_TIME$ since $OBT_START$ that is time bin centered (i.e. with an integration 
%time of 4s, $RELATIVE_TIME$ will 2-6-10…, as opposed to 0-4-8…). It is possible that 
% $RELATIVE_TIME$ “jumps”, e.g. when the attenuator is moving in and the affected time bin 
% becomes undetermined and is left out. Additionally, there is an energy channel reference 
% CHANNEL, with indices into the energy axis in the second extension. Given below is an I
% DL-based structure notation.
\item Background monitor light curves. Background monitor light curves are similar to the Quick-look light curves but only
counts recorded by the background detector pixels are included, and the integration time is 8 seconds.
\item Variance data are onboard computed variance of 40 successive detector-summed count rates
based on 0.1 sec integration.
\item Quick-look spectra, which are energy snapshots of energy spectra (32 science energy bins) for
each of the 32 detectors with 32 sec exposure time.
STIX takes a snapshot of energy spectra every 1024
seconds in the nominal observation mode.
\item Calibration spectra. Calibration spectra are accumulated for events from 128 weak onboard $^{133}$Ba radioactive sources 
(a total activity of 4500 Bq) 
 when the solar count rate is small compared to the background.
Identification of such quiet periods is done autonomously, based on the presence of a
TC-specified time gap between two successive photons.  
ADC readouts of each channel are accumulated in individual spectrum.
After an accumulation a long period (typically 24 hours),  the spectra are 
formmated and redirected to in the quick-look low-latency data.
\end{itemize}
\subsection{Diagnostic and event data}
When a failure is detected directly by STIX, the instrument’s response includes sending a
“dedicated” telemetry event report with appropriate diagnostic information included.

\subsection{Bulk science data}

Bulk science data are different combinations of summing and compression of the basic pixel data stored in STIX onboard archive memory, which can be used for spectrocopy and imaging.
To cope with the limited available telemetry, they are not automatically included in the telemetry.
Onboard formation of the science data is invoked by data request telecommands.
Each science request can select subsets of energies, pixels and detectors.
Six different science data can be generated:
\begin{itemize}
 \item Level-0 X-ray data is the least processed data and contains uncompressed counts for the selected energies, pixels and detectors;
\item Level-1 is essentially the same as Level-0 but the counts are compressed onboard before being sent to the ground;
\item Level-2 data are further compressed counts from the L1 pixel data, in which are summed down to 4 before compression.
\item Level-3 data are visibilities, which further reduces the data by combining the four summed pixel counts into complex visibility which is also compressed.
\item Level-4 data are detector summed spectrograms;
\item High-time resolution aspect data.
\end{itemize}
It should be noticed that the data levels described 
above are on-board data levels. They are sent to the ground in the format of raw binary packets and 
considered as level-0 at STIX data center.  
\begin{table}[h]
\centering
\caption{STIX raw telemetry data coverage, data rate and typical reception delay at SDC.  }
\resizebox{0.8\linewidth}{!}{
\begin{tabular}{lllll}
\hline
Category &  Coverage &Daily data volume & Reception Delay  \\ \hline
Housekeeping           & continous &   143 kiB  & hours to days \\ \hline
Quicklook     & continous         & $\sim$ 358 kiB  & hours to days   \\ \hline

Science     & ground-selected only &  $\sim$ MiB to $\sim$ 10 MiB &  2 to 12 weeks   \\ \hline
Calibration  & quiet sun periods  & 100 kiB  & $\sim$ 1 day \\ \hline
\end{tabular}
}
\label{tb:raw_types}
\end{table}

Housekeeping data, quick-look data and calibration data are directed to the
common low latency data stored in SSMM in the spacecraft.
The coverage, data rate and latency of 
different types of STIX raw telemetry data are
summarized in Table \ref{tb:raw_types}.

%https://issues.cosmos.esa.int/solarorbiterwiki/display/SOSP/SOC+Archive+Plan?preview=%2F11734040%2F21759347%2FSOL-SGS-PL-0009-SOARPL-2.0.pdf

\section{Data reception}

During nominal science operations,
Low latency data are down-linked in the very next ground station pass regardless of orbital geometry, 
whereas science data are only down-linked when the bandwidth permits.
The downloaded instrument telemetry data are first processed by ESA's ground segment
software at the Solar Orbiter mission control center. Then they
are distributed to instrument teams by the ESA EGOS Data
Dissemination System (EDDS) (\cite{EDDS}) according instrument teams' preset conditions.

Telemetry data received by STIX data center from EDDS  are in the binary format, and have the same
 information content as the original telemetry generated by STIX.
The low-latency data arrives at STIX data center with delays ranging from a few minutes to a few days, depending on whether there
are antenna passes, whereas science data may arrive several weeks after being generated onboard.

In addition to the telemetry data, STIX data center also receives SPICE kernels (\cite{spice1996,spice2018})  from the science operations center.
They contain information of spacecraft ephemeris and clocking calibration factors required for time conversions.


\section{Data processing pipelines}
\subsection{Data processing pipeline overview}


\begin{figure*}
    \centering
    \includegraphics[width=0.9\textwidth]{figures/pipelines.pdf}
    \caption{Data processing pipelines at STIX data center.}
    \label{fig:main_pipelines}
\end{figure*}
New telemetry data arriving at STIX data centers is immediately processed by the pipelines as shown 
in Fig. \ref{fig:main_pipelines}. 
The processing is started from raw packet parsing. Parsed level-1 packets are written to a NoSQL database. 
Then they are selected  are processed in 4 different paths, 
In the first path,  housekeeping, quick-look and science packets 
are selected and used to create L1-A FITS files, 
which can be used for scientific analysis. 
The second path selects calibration data from the database and extracts energy calibration factors for 
instrument monitoring. In the third path, quick-look packets are selected and used to identify solar
flares; The fourth path processes flaring data, 
generating higher data products.   

\subsection{Raw data parsing and database}
New telemetry data arrived at STIX data center are immediately processed by STIX data parser, 
which is capable of parsing all types of STIX binary telecommands and telemetry packets.   
Parsing of packets is based on the mission interface database (MIB), which 
contains information on packet parameters such as names, descriptions, lengths,
data types. The parser extracts header and parameters for each raw binary packet
using the information in the MIB.
Packets after parsing contain raw values of parameters. 
 Spacecraft clock times are further converted to UTC times using 
the latest version of SPICE kernel data (\cite{spice1996,spice2018}), compressed integers are  
decompressed by using a LUT table and 
housekeeping raw values are converted to 
human-readable values using the calibration factors or look-up tables stored in the MIB. 
Packets after the above processing steps are called level-1 packets. They
have tree structures, each containing two nodes "header" and "parameters";
the former contains the basic information  of the packet such as timestamp, packet type,
and the latter contains names, 
raw values, engineering values (or decompressed integers) and child nodes of a list of parameters.
Level-1 packets are written 
to a NoSQL database. 
NoSQL database is schemaless,  making it ideal to store data with complex
structures but small sizes like STIX level-1 packets.  
NoSQL is also 

In addition to packets, the NoSQL also stores other data extracted by the parser during packets parsing, 
for example, 
\begin{itemize}
  \item Raw file metadata. contains filename, reception time, observation start time and stop time, MIB and SPICE kernels used by the parser ; 
  \item Bulk science metadata, which contains the observation time range, detector and pixels masks, energy ranges, 
   bulk science data types, and IDs of packets in the database; 
  \item Quick-look light curves.  The dataset contains data of counts, time range, energy bins extracted from QL LC packets, 
     allowing for  quick access to QL data with web pages or APIs;
  \item STIX instrument configurations. 
  It contains the instrument configuration parameters for such as  energy conversion factors and AISICs, 
  allowing fast tracking of the instrument settings;
\end{itemize}
The information are necessary for subsequent processing. The
database is  accessible through web pages at STIX data center website or python APIs. 
\subsection{FITS file creation}
\subsubsection{Level-1A FITS files}
After each new raw telemetry file is being parsed, 
housekeeping, quick-look, and science packets  
are selected from the NoSQL database successively 
and checked for data integrity and consistency. 
Then packets of the same type are merged for creations of pre-release of 
level-1 data products (Level-1A)
 in the FITS format (\cite{fits}), 
which is a portable file standard widely used in the astronomy 
community to store images and tables.
It should be noted that the data levels mentioning here are data processing levels. 
In order not to confuse two conventions, we use different notations to indicate onboard level in this paper. 
Metadata such as observation time range, creation time, filenames,  checksum are written to 
the primary header data units (HDU) of fits files. 
They are also written to a dataset in the database.

\subsubsection{Level-1 FITS files}
Level-1A FITS files are generated automatically and available at STIX data center within minutes 
after the reception of a raw file.
It should be noticed that predicted SPICE kernels may be used when the reconstructed spice kernels  
are not available, and the L1A data are subjected to some known issues 
due to bugs in the early version of the flight software.
After all resources are validated,  Level-1  FITS files are created again. 
Level-1  FITS products have almost the same data structures as Level-1A.

\subsubsection{Level-2 FITS files}


\subsection{Energy calibration}
STIX converts energies from ADC units to keV onboard. This relies on
an energy look-up table (ELUT) regularly updated  with telecommands.
An ELUT can be constructed using energy conversion factors obtained by studying
 positions of photo peaks in the calibration spectra measured for the onboard Ba$^{133}$ sources.  
 As energy conversion factors may change with temperature 
 or polarization effects,  calibration spectra need to be analyzed promptly on the ground. 


\begin{figure}
 \centering
  \includegraphics[width=0.8\linewidth]{figures/cal-fit.pdf}
  \caption{An example of STIX in-flight calibration count spectrum.
  The three strongest lines, from left to right, are from 31 keV, 35 keV, and 81 keV
  photons. The first two peaks are fitted by the double-Gaussian function and the high energy peak by  
  the crystal-ball function. }
    \label{fig:cal-fit}
\end{figure}
The right panel of Fig.~\ref{fig:cal-fit} shows 
an example of STIX count spectrum from an in-flight calibration run.  
The three strongest lines are from 31 keV, 35 keV and 81 keV photons. 
The first two peaks are fitted with the double-Gaussian function and the third peak
by the crystall-ball function \cite{crsystallball},  which consists of a Gaussian core portion 
and a power-law low-end tail, below a certain threshold.
Then a linear line is fitted to the relation between
the three fitted peak positions (in ADC unit) and the photon energies in units of keV.  
The interception and slope indicate the baseline and gain of the channel. 
The results were compared to those determined using the ECC method (see \cite{ecc,ecc2})), 
 which were found to be consistent within errors.
The above steps are performed for each calibration run automatically 
once the data is available at STIX data center.  The results are written to a
 dataset in  NoSQL database.  It is accessible through a web page at STIX data center or python APIs.

Calibration factors are monitored continuously. Once significant changes are 
observed from those used for the construction of  the onboard ELUT, 
 a new ELUT is  created and  then uploaded to STIX after validation by the operations team. 

\subsection{Solar flare identification}
The in-flight software identifies solar flares by comparing  the count rate of 
the background detector with that of other detectors 
 and packs the results into the QL flare flag and location reports.
However,  the reports only provide limited information on flares due to the constraints of telemetry 
bandwidth and the limitation of onboard computing resources, and micro-flares are not 
reported due to the relative high trigger threshold.
Using QL light curves, solar flares can be identified on ground in great detail.

QL light curves of the energy range from  4 to 10 keV are used for solar flare identification
on the ground. Identification is based on the fact 
that the background event rate in that energy range is almost constant 
over a timescale of days during quiet sun periods. 
The procedure involves the following steps:
\begin{itemize}
  \item Light curve smoothing. The selected light curve is filtered using an unweighted
  moving average filter with a time window of 1 minute to smooth out statistical fluctuations and electric surge spikes;
  \item Identification of flare peaks. Peaks with counts exceeding a threshold of $2\sigma$ above the background level
   are selected. The duration of a peak is given by the time difference between the first 
   crossing of the threshold on the left and right sides of the peak;  
  \item Merging of flare peaks. If the time difference between the two peaks is less than 5 minutes,
   they are considered to be from the same flare.
\end{itemize}

\begin{figure}
  \centering
  \includegraphics[width=0.8\linewidth]{figures/flaredet.pdf}
  \caption{STIX 4 -- 10 keV QL light curve recorded from  2022-02-13T21:00:00Z to 2022-02-14T03:00:00Z and 
  identified flares.   The light curve was smoothed using a moving average filter with a time window
  of 1 minute. The identified peaks are marked with plus signs, and flare time ranges are highlighted in yellow.
  }
  \label{fig:flare-det}
\end{figure}
As an example, Fig.~\ref{fig:flare-det} shows  the 4 -- 10 keV QL  light curve recorded from 
2022-02-13T21:00:00Z to 2022-02-14T03:00:00Z, as well as the smoothed light curve and  identified solar flares (marked with plus signs).

For each identified solar flare,  start time, end time, peak time and peak 
counts are extracted from the selected light curve and assigned 
an 8-digit identification number of the format yymmddHHMM, which represents the solar flare peak time. 
For example, the identification number 2201010000  indicates that 
the solar flare peak time is at 2022-01-01T00:00:00 UT. 
The above steps are repeated  for the QL light curves of other four higher energies
in the same time frame. This can provide information on the upper limit of the X-ray energy produced by the flare,
which is used to optimise the selection of scientific data to be downloaded. 
Then all extracted information as well as the ephemeris data calculated for the peak time are 
written to a dataset called solar flare list in the NoSQL database.
The flare list database can be queried using web interface or stixpy.
\subsection{Determination of background level}
As we mentioned earlier, a threshold value calculated using background data needs to be provided when
performing flare identification.  
The background, although stable within a few days, can change over long periods of time,
 a suitable threshold is important for the accuracy of the identification. 
 Therefore, QL light curves for  quiet sun periods are selected by excluding flaring periods. 
 Then median values and variances are calculated from the selected light curves and 
 written to a dataset in the database. They are used as input for the next flare identification. 
 Those processing steps are performed automatically after flare identification.

\subsection{Solar flare data analysis pipeline}
\subsubsection{Solar flare GOES classes}
\begin{figure}
  \centering
  \includegraphics[width=0.8\linewidth]{figures/goes_stix_flux_paper.pdf}
  \caption{Scatter plot of GOES low channel peak flux with respect to STIX 1-AU equivalent  peak count rate in the 4 -- 10 keV range
  for 717 solar flares observed by both GOES and STIX duration the commissioning phase. 
  The solid line is a linear fit to the log-log graph. 
  The slope and 
  interception for the best fit are $0.622\pm 0.002$ and $-7.376\pm0.067$, respectively.
STIX count rate are background subtracted and corrected for the distance variations between the Sun and Solar Orbiter. 
}
\label{fig:goes-stix}
\end{figure}
It is straightforward to calculate the GOES
class of a flare observed by STIX from GOES flux if it is observed by GOES. 
However,  most of the time Solar Orbiter is far away from Earth and looks at 
the sun from different angles. Therefore a considerable number of flares observed by STIX
 are not observed by near earth satellites (and vice verse). 
As already discussed in Ref.~\cite{andrea2021}, 
it is possible to estimate GOES classes by using STIX count rates.
In order to study the correction, we selected 717 solar flares observed by 
by both GOES  and   STIX during the commissioning phase.   
Fig.~\ref{fig:goes-stix} shows the scatter plot of GOES low channel peak flux with respect to 
STIX peak count rate  of the 4 -- 10 keV QL light curves. 
STIX count rates  have been subtracted for background,  and corrected for 
the difference distance of Solar Orbiter to the Sun using $c'=c r^2$, where $c$ is the count rate after background subtraction
 and $r$ the distance between Solar Orbiter and the Sun. 
As can be seen in the figure, there is clear correlation between
these two quantities.  The wide spread at the low flux could be explained by the difference in 
the energy response of the two instruments and in the flare temperatures.
The correlation can be fitted with a linear line in the log-log scale: 
\begin{equation}
f=10^{a+b\log_{10} c'}, 
\label{eq:goes-stix}
\end{equation}
where $a$ and $b$ are two free parameters, $f$ is the GOES flux  in units of W/m$^2$.
 The best fit was obtained at $a=0.622\pm 0.002$ and $b=-7.376\pm0.067$.
The obtained model is used to estimate  GOES classes 
 of flares that observed by STIX. 
The results as well as those determined from GOES measurements
are attached to  the flare entry in the flare list database. 
\subsubsection{Coarse flare location}
STIX uses a dedicated sub-collimator called Coarse Flare Locator (CFL) to estimate 
rough location of flares.  The CFL consists of a front grid with
a distinctive pattern which selectively illuminates pixels of a 
dedicated detector based on the source location.
Flare location is estimated onboard by maximizing
the correlation of counts in the pixels of pixels and mean counts from other detectors   with expectations 
calculated for an array of 65$\times$65 possible flare locations.
However, due to the constraints of onboard computation, the onboard flare location is  only calculated for intermediate flares 
and has an accuracy of about 2 arcmin. 
With the downloaded measured counts  in each pixel,
the coarse flare location can be reconstructed to on the ground as well. 
This allows for more sophisticated algorithms, greater flexibility of selecting time and energy 
range to be integrated, and more careful background subtraction.

\begin{figure*}
  \centering
  \includegraphics[width=0.95\linewidth]{figures/cflMay07.pdf}
  \caption{
   Left: Area of illuminated regions of 12 pixels calculated by combining
  the twelve pixels counts with averages from other detectors, and the best fit.
   Middle: solution of flare location as seen by Solar Orbiter for STIX flare  2105071900.
   The best fit centroid location is obtained at (550, 278) arcsec. 
   The 1-$\sigma$, 2-$\sigma$ and 3-$\sigma$ confidence contours  are shown.
    Right: 
  simulated shadow pattern (the grey shaded regions) of CFL sub-collimator on STIX detector 
  for STIX flare 2105071900. 
    }
  \label{fig:cflpattern}
\end{figure*}

Estimation of the coarse location of a flare on ground is performed
for each flare after the pixel data (onboard L1) are . 
It includes the following steps:
\begin{itemize}
  \item Selection of pixel data based on the flare time range and energy range information stored in the flare list dataset;
  \item Subtraction of background. The most recent level-1 dataset for the quiet sun period is used for background subtraction;
  \item Calculation of the mean fluence except for the CFL and the background detectors; 
      Except for the those two detectors, the ratio between the illuminated area and total area $r$ only depends on the 
      slit width and the pitch width. It doesn't vary with the position of the flare in the STIX FOV.  
      Therefore, the fluence of a detector can be given by $F=c/(r*S_d)$,  where $c$ is the registered 
      count and $S_\textrm{det}$ the sum of the sensitive areas of twelve pixels. 
  \item Calculation of illuminated areas of CFL pixels $\vec{S}$ using $\vec{S} = \vec{C}/F$, where $C$  is an array of counts 
  recorded by 12 pixels. Their errors of the areas are also calculated; 
  \item  Coarse flare location estimation is performed by minizi

  %%\item 
\end{itemize}




\begin{figure*}
  \centering
  \includegraphics[width=0.7\linewidth]{figures/cflError.pdf}
  \caption{Top: reconstructed coarse flare location solutions against actual locations along the x-axis (left) and y-axis (right). Bottom:
  Differences between the reconstructed and randomly generated locations along
   the x-axis (left) and y-axis (right). The generated sources are assumed to be point-like. Note that statistical
   uncertainties are not considered here. 
    }
  \label{fig:cflerror}
 %/FHNW/STIX/SolarFlareAnalysis/stix_simulator/cfl
\end{figure*}

\section{STIX data products}



The Level0 archive contains TM which has been parsed or decommuated into readable structures but no additional external information is include:

times are not converted to UTC
no calibration or conversions applied
for STIX we need to decide if we decompress / combine X-ray L0 the count/trigger data at this stage or in the next level L1


\subsection{Auxilary data}

\begin{table}
\centering
\caption{Level 1 data products}
\begin{tabular}{llll}
Category & Type   &  Naming convention  & Remarks   \\ \hline
 Housekeeping & hk\_mini  &  & Houskeeping in BOOT mode   \\
 & hk\_maxi  &  & houskeeping data in NOMINAL   \\
 Quick-look &  light curves &  & Quicklook lightcurves \\
  &  variance &  & variance \\
  &  spectra &  &  \\
  &  background monitor &  &  \\
    &  flare location &  &  \\
 Calibration data &   &  &  \\
\end{tabular}
\end{table}

\section{Data selection and automation}
To wisely allocate the limited telemetry available to STIX,
the STIX instruments first sends only QuickLook (QL) data at
low resolution to the ground (for details see Krucker et al. 2020).
Figure 3 shows the STIX QL lightcurve at the lowest energy
channel (4-10 keV) during the time period from June 4 through
June 14. From the QL data, the STIX team then selects the sci-
entifically most promising flares, and so-called data requests are
sent to STIX to download the selected flares at the chosen res-
olution.
As the foregoing discussion suggests,
 there are three important distinctions between the 
 QL and primary X-ray data handling: first, the QL data represent only 
 spatially-integrated data;  second, the QL data are acquired and 
 transmitted with fixed parameters such as a 4 s cadence; third, there is no 
 provision for onboard QL data selection. 
In contrast, the much more voluminous primary data can only 
be transmitted on a highly selective (flare-associated) basis, albeit with a wide range of choices for time and energy
 resolution, and coverage. These choices can be optimized to match the dynamic characteristics of the flare in question.  
  Thus by limiting consideration to flare intervals only, and by optimizing time and energy range and 
 resolution for the individual flare, the scientific efficiency of imaging and spectroscopic data telemetry can be greatly enhanced.


There are two approaches to selecting the time- and energy
resolution and range for the bulk spectroscopic and imaging data.
The first approach relies on the ground segment to select param
eters for the downlink of flare T/M. This selection can be based
on the QL light curves in 5 different energy bands which together
provide a robust indication of flare timing, intensity and spectral
properties. The ground segment can use this to choose optimized
time range, energy range, and resolution for different phases of
the flare, predict the resulting T/M volume and upload corresponding analysis requests.
The onboard software then applies
these parameters to the data stored in the archive buffer. This
option is viable because the latency of the QL data (up to 24 h)
is much shorter than the multi-week longevity of the archive
buffer. This provides time for the ground segment to make selections 
and upload parameters and for the FSW to process the data.
Since each ground system requests result in a predictable telemetry volume, 
the STIX ground system is responsible for complying with T/M corridor restrictions on the data rate.

There are three downsides to this approach to data selection:
first, the selections and choices are a labor-intensive activity;
second, selections must be made and satisfied before the onboard
archive is overwritten and so that the choices must be made in a 
timely fashion; third, there is an additional operational latency of
at least a few days between flare occurrence and the transmission
of the requested X-ray imaging and spectroscopic data.
With the second approach (“autonomy”), the FSW uses the
output of the onboard flare detection algorithm to identify rele
vant time ranges. It then uses automatic, parametrized algorithms
to optimize the time and energy resolution for imaging and spec-
troscopic data for inclusion in bulk T/M. As of this writing, this
option has not yet been fully implemented. When using autonomy,
the QL telemetry (Table 3) will include flare and telemetry
management information to show the analysis status of recent
events.

\begin{itemize}
\item L1 request for STIX solar flare.\\
For detected flares,  a script is used to create data requests.
If the background subtracted peak count rate of a flare is greater 150,
both L1 and spectrogram requests are created;  For micro-flares, as only
spectrograms are request normally as the statistics is too
low to reconstruct flare images.  Aspect data requests
and some extra data requests are created for events of interest in the
STIX operations team.
For external users, data request forms can be also submitted using a web tool
at the data center.
\item Background L1 data request.
\item Continous data request Requests for background data
\end{itemize}

An unique ID is assigned to each data request automatically.
The ID naming convention is yyddmmxxxx, in which yyddmm indicate the observation year (without century), month, date,
and the last four digits indicates the serial number of the data request in the day.
IDs are used to track the status of data requests, and reterieve data products from the
STIX data center.
The information of created data requests are stored in the NoSQL database on the same server.
They are converted to instrument operation requests (IORs) after a series of checks.
Requested data will arrive at STIX data center within a time frame of two weeks to three months, depending on
telemetry allocations.

\section{Database}
\subsection{Raw data packet database} 
\subsection{Configuration database}
\section{Web data browsers}
\subsection{Quick-look light curve}
\subsection{Science data quick analysis}
\subsubsection{Calibration data}

calibration data products
https://fermi.gsfc.nasa.gov/ssc/data/access/gbm/
\subsubsection{Solar Orbiter orbit viewer}
The ISDC maintains a web page through which
data, software, its newletter, general information and links
to other sites are provided. This is accessible through
http://isdc.unige.ch

When a new data file from the platform is received at the PPDC,
it triggers an autonomous start of the dedicated program that decodes and
interprets its contents. The binary data contain the spacecraft location, attitude, speed, and GPS timestamps with increments every half second. The GPS timestamps are converted into Unix-timestamps, where the leap seconds are also considered. After processing, the platform data are written to the ROOT format files. The data start and stop time, data processing time, input filename and ID of the output file of each processing are recorded in a dedicated database table.
SPICE kernel

Updated once per day.

At the center of the Sun.
It is worth mentioning that has to corrected for.
This can be done by using the web tool provided at the auxiliary data center at


\section{Tools for scientitic analysis }
\subsection{Data access and APIs}
stixdcpy allows you to query and download data which are available at STIX data center, include
Quick-look light curves
Housekeeping data
Science data
Energy calibration dat which are later included to the low-latency telemetry stream.
Auxilary data
STIX solar flare list
\subsection{Spectral analysis software}
\subsection{STIX Imagging software}



\section{Summary}
\label{sec:summary}



% WARNING
%-------------------------------------------------------------------
% Please note that we have included the references to the file aa.dem in
% order to compile it, but we ask you to:
%
% - use BibTeX with the regular commands:
%   \bibliographystyle{aa} % style aa.bst
%   \bibliography{Yourfile} % your references Yourfile.bib
%
% - join the .bib files when you upload your source files
%-------------------------------------------------------------------

\bibliographystyle{aa}
\bibliography{citations}

\end{document}
