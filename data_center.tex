%
\documentclass[referee]{aa} % for a referee version
%\documentclass[onecolumn]{aa} % for a paper on 1 column  
%\documentclass[longauth]{aa} % for the long lists of affiliations 
%\documentclass[letter]{aa} % for the letters 
%\documentclass[bibyear]{aa} % if the references are not structured 
%                              according to the author-year natbib style


%\documentclass{aa}  
\usepackage{hyperref}
\usepackage{graphicx}
%%%%%%%%%%%%%%%%%%%%%%%%%%%%%%%%%%%%%%%%
\usepackage{txfonts}
%\usepackage{multirow}
%%%%%%%%%%%%%%%%%%%%%%%%%%%%%%%%%%%%%%%%
\usepackage{subcaption}
\usepackage{flushend}
  % results heading (mandatory)
\begin{document} 


   \title{Data processing pipelines and the data platform for the X-ray spectrometer/imager STIX onboard Solar Orbiter}

   \subtitle{}

   \author{Hualin Xiao
          \inst{1}
          \and 
          Shane Maloney 
          \inst{2}
          \and S\"am Krucker\inst{1}
          \and Paolo Massa \inst{6}
          \and Lastufka Erica \inst{1}
              \and Ewan Dickson \inst{4}
          \and Andrea Francesco Battaglia\inst{3}
          \and Frederic Schuller \inst{7}
       %   \and André Csillaghy\inst{1} 
          \and Ryan Daniel \inst{1}
          \and László Etesi \inst{1}
          \and Nicky Hochmuth \inst{1}
          \and Collier Hannah \inst{1}
      %    \and Gordon Hurford\inst{1} 
          \and Olivier Limousin \inst{5}
     %     \and Astrid M. Veronig\inst{4} 
      %%    \and Alexander Warmuth \inst{7}
          \and other STIX data center contributors
         }

   \institute{University of Applied Sciences and Arts Northwestern Switzerland (FHNW), 5200 Windisch, Switzerland \\
              \email{hualin.xiao@fhnw.ch}
         \and
          Astrophysics Research Group, School of Physics, Trinity College Dublin, Dublin 2, Ireland
          \and
             ETH Z\"urich, R\"amistrasse 101, 8092 Z\"urich, Switzerland
         \and University of Graz, Universitätspl. 3, 8010 Graz, Austria
         \and IRFU, CEA, Université Paris-Saclay, and Université Paris Diderot, AIM, Sorbonne Paris Cité, CEA, CNRS, 91191 Gif-sur-Yvette,
         France
         \and Dipartimento di Matematica, Università degli Studi di Genova, Via Dodecaneso 35, 16146 Genova, Italy
         \and Leibniz-Institut für Astrophysik Potsdam (AIP), An der Sternwarte 16, D-14482 Potsdam, Germany
             }

   \date{\today}

% \abstract{}{}{}{}{} 
% 5 {} token are mandatory
 
  \abstract
  % context heading (optional)
   {The Spectrometer/Telescope for Imaging X-rays (STIX) instrument is one 
   of the ten instruments onboard the Solar Orbiter to measure spectra and take images of solar flare X-rays in the energy range of 4 to 150 keV over a wide range of sizes.} %leave it empty if necessary  
  % {context.}
  % aims heading (mandatory)
   {During nominal operation, STIX continuously generates data. Constant data flow requires fully automated data processing pipelines to process,
     analyze the data and a data platform to manage, 
   visualize and distribute STIX generated by the pipelines.   
   }
   {
   A data center has been established at FHNW. 
   It consists of automated pipelines and a data platform.
   The pipelines process telemetry data, perform standard scientific analysis, 
    and generate data products at different levels.  
   The software running on the platform 
   consists of databases and web-based applications. The platform provides application interfaces for STIX data users.  
   The pipelines generate telemetry at different levels and perform standard scientific analysis. 
   }
   {
 The platform provides  all STIX data products of different levels and also provides users 
 with various web-based tools to search for,  browser STIX data products. 
 It also provides web-based tools to perform common analysis tasks with STIX data. 
  The data center is designed to work in a fully automatic mode with minimal human intervention. The concept has proven successful 
 and has been running continuously for more than two years. The platform not only facilitates the operations of the instrument but also provides great support to STIX data users.}
 

   \keywords{Solar flares -- Data platform --
                STIX data products --
                -- X-ray imaging, 
                Data processing pipeline
               }
  \titlerunning{STIX data center}
  \authorrunning{Hualin Xiao and STIX team}
   \maketitle

%-------------------------------------------------------------------

\section{Introduction}
Solar Orbiter is a Sun-observing mission of the European Space Agency that 
addresses the interaction between the Sun and the heliosphere.
It was launched on Feb. 10, 2020, for a nominal mission duration of seven years and a planned 
extension of three years. It carries ten sets of instruments for comprehensive
remote-sensing and in-situ measurements. 
Solar Orbiter will perform detailed measurements of the Sun as close as 0.28 AU and for the first time look at its uncharted polar regions (\cite{SolarOrbiter2020}).  
The Spectrometer Telescope for Imaging X-rays (STIX) is one of the ten instruments onboard the Solar Orbiter.  
It measures X-rays from 4 to 150 keV and takes X-ray images with a few arcsec angular resolutions by using an indirect imaging technique,
based on the Moiré effect.  STIX consists of 32 collimators with grids and pixelated cadmium telluride detector unit Caliste-SO. 
The main science objective of STIX is to study the extremely hot solar plasma and the high-energy electrons accelerated during solar flares. STIX provides intensity,  spectrum, timing, and location of accelerated electron information of solar flares.
For more information on STIX instrumentation and its scientific capabilities, we refer to the instrumentation paper (\cite{StixInstrument}).


STIX continuously acquires data during its nominal operation. 
In order to reduce the telemetry to be down-linked, STIX compresses and formats data into different types of telemetry packets onboard.
 The total number of telemetry packets reaches hundreds. 
Being aware of the complexity of STIX data analysis and the need of bringing the data to the 
solar physics community, automated data processing pipelines and a data platform have been developed at FHNW. 
The pipelines turn raw telemetry data into processed information,  and produce data products at different levels
 that can be used for scientific analysis.
 The platform stores all data products and their metadata. 
 It also provides various web-based tools which facilitate instrument operations,
  access and use of STIX data for research activities.

 The purpose of this paper is to describe the processing pipelines,
 the core algorithms of standard analysis, the STIX data products, and the tools provided by the 
 STIX data platform for STIX data users.
This paper is organized as follows. Section \ref{sec:raw-data} briefly introduces STIX raw telemetry types, data flow and
telemetry processing pipelines. 
In the end, it is a summary \ref{sec:summary}.
\section{STIX raw telemetry data}
\label{sec:raw-data}
When STIX  is in the nominal observation mode, STIX continuously collects energy deposition information from 32 detector units, the aspect system,
and engineering sensors.
After being processed by the onboard FPGA and CPU,  
low-latency telemetry data are directed to the
storage in the spacecraft,  whereas high-time resolution pixel data are written to STIX onboard archive memory for 
later generation of science data as requests from the ground. 
STIX generates hundreds of different types of telemetry data. 
However, from the data user's perspective, 
there are tens of most important data types, which can be classified into three categories:
 housekeeping data, quick-look data, and science data.
\subsection{Housekeeping data}
Housekeeping (HK) data are generated as long as STIX is powered on.
A housekeeping packet contains information on the instrument,  such as temperature,
voltage, current, the status of switches, CPU usage, onboard flash memory usage, 
detector median and maximum trigger rates, and readouts from the four photodiodes.
During nominal operation, STIX generates a housekeeping every 64 seconds, which 
results in daily raw telemetry of  143 KiB.
Housekeeping data not only provide instrument status information, but also 
the pointing information, which can be derived from the the aspect system readouts. 
Housekeeping data are down-linked to the ground with the highest priority and are available on the platform
within 24 hours. 

\subsection{Quick-look data}
STIX generates four different types of quick-look data in NOMINAL mode (regular science mode):
\begin{itemize}
\item Light curve data contain time series 
of detector-summed counts (Counts recorded for the two 
special detectors  the background monitor and coarse flare detector are excluded) 
in five different energy bands, rate control regions and detector-summed triggers 
integrated over 4 seconds. Note that counts are not corrected for dead time,  transmission, impacts of the presence of the attenuator, 
or rate control regimes.

\item Background data. 
The BKG subcollimator consists of an open front grid window
and a rear grid window that is fully opaque except for six small
apertures. It is used to measure background and incident flux when other detectors are saturated.  
Background data contain counts and trigger sum over BKG pixels and integrated over 8s in the five energy
bands the same as those of QL light curves. 
\item Variance data are the onboard computed variance of 40 successive detector-summed count rates
based on 0.1-second integration.
\item Quick-look spectra. Quick-look spectra 
 are energy snapshots of energy spectra (32 science energy bins) for
each of the 32 detectors with 32 s exposure time.
STIX takes a snapshot of energy spectra every 1024
seconds during nominal operation.
\item Calibration data. 
Calibration data contain high-resolution raw energy spectra of each pixel accumulated for 
events emitted by 128 weak $^{133}$Ba radioactive sources. 
They are used to determine the energy conversion factors of each pixel by using photopeaks in the calibration spectra.  
During nominal operations, a calibration run typically lasts about 24 hours, and a new calibration run is started immediately. 

\end{itemize}


\subsection{Bulk science data}

Bulk science data are different combinations of summing and compression of the raw pixel data stored in STIX 
onboard archive memory, which are only generated after receiving data request commands from the ground. 
STIX generates six different types of bulk science data: 
\begin{itemize}
 \item Raw pixel data: Raw pixel data are the least processed data and contain
  uncompressed counts for the selected energies, pixels, and detectors; 
\item Pixel data are essentially the same as raw pixel data, 
but the counts are compressed onboard before being sent to the spacecraft;
\item Compressed pixel data are further compressed counts from the pixel data,
 in which are summed down to 4 before compression.
\item Visibilities further reduces the data by combining the four summed pixel counts into complex visibility, 
which is also compressed.
\item Spectrograms are detector summed spectrograms; 
\item High-time resolution readouts of photodiodes in the aspect system;
\end{itemize}
Bulk science data are sent to the ground in the format of raw binary packets and  are
considered level-0 at the STIX data center.  
%\begin{table}[ht]
%\centering
%\caption{Coverage of STIX raw telemetry data, data rate, and typical reception delay at SDC.  }
%\resizebox{1.0\linewidth}{!}{
%\begin{tabular}{lllll}
%\hline
%Category &  coverage &Daily data volume & Latency\\ \hline
%Housekeeping           & continuous &   143 kiB  & hours to days \\ \hline
%Quicklook     & continuous         & $\sim$ 358 kiB  & hours to days   \\ \hline

%Science     & ground-selected only &  $\sim$ MiB to $\sim$ 10 MiB &  2 to 12 weeks   \\ \hline
%Calibration  & quiet sun periods  & 100 kiB  & $\sim$ 1 day \\ \hline
%\end{tabular}
%}
%\label{tb:raw_types}
%\end{table}

%The coverage, data rate, and latency of 
%different types of STIX raw telemetry data are
%summarized in Table \ref{tb:raw_types}.
%https://issues.cosmos.esa.int/solarorbiterwiki/display/SOSP/SOC+Archive+Plan?preview=%2F11734040%2F21759347%2FSOL-SGS-PL-0009-SOARPL-2.0.pdf

\section{Data reception}

During nominal science operations, low-latency data (housekeeping data, quick-look data and calibration data)
are down-linked to 
the very next ground station pass regardless of orbital geometry, 
whereas science data are only down-linked when the bandwidth permits.
Telemetry data received by ground stations are first processed by the ESA's ground segment
software at the Solar Orbiter mission control center. 
Then they are distributed to instrument teams regularly via the ESA EGOS Data Dissemination System (EDDS) (\cite{EDDS}). 

Low-latency data arrive STIX data servers with a  delay ranging from 
a few minutes to a few days, depending on whether it is in a pass of the ground station. 
Science data may arrive several weeks to a few months after being generated onboard.

Apart from instrument telemetry data, the STIX data servers also receive
  auxiliary (\cite{spice1996,spice2018})  from the science operations center,
which contain information on spacecraft ephemeris, attitude and  
calibration factors required for the conversion of onboard timestamps 
to commonly used timestamps. 


\section{Data processing pipelines}
\subsection{Data processing pipeline overview}


\begin{figure*}
    \centering
    \includegraphics[width=0.9\textwidth]{figures/pipelines.pdf}
    \caption{Telemetry processing pipelines at STIX data center.}
    \label{fig:main_pipelines}
\end{figure*}
New telemetry data arrive at the STIX data center are processed immediately by the data processing pipeline  
as shown in Fig. \ref{fig:main_pipelines}. 
It starts with level-1 processing of raw packets, which includes several steps such as  packet parsing, 
integer decompressing and timestamp converting. 
Packets after the processing is at the L1 level, they are written to a NoSQL database. 
Then they are selected and processed in four different paths. 
In the first path,  housekeeping, quick-look and science data are successively selected and 
used to create level-1 FITS files. The second path selects calibration data for the determination
 of energy calibration factors. 
 In the third path, quick-look packets are selected and used to identify solar
flares; The fourth path performs standard analysis of science data for flares.

\subsection{Raw packet parsing and database}
Telemetry data that arrived at STIX data center are binary packets. 
Each packet contains a fixed-length header 
and a list of parameters that vary with the type of packet. 
The parsing of parameters is based on the information in 
the mission interface database (MIB), which contains the name and the 
length of each parameter for each type of packet. 
Packets after parsing contain raw values of parameters, 
which need converting to physical values. 
Raw values of spacecraft-local times are converted to UTC times by using 
the latest version of SPICE kernels (\cite{spice1996,spice2018});
Raw values of housekeeping parameters are converted to physical values using 
the pre-calibrated conversion factors in the MIB. For compressed counts, 
they are decompressed using a look-up table. 

The raw and physical values of parameters of each packet as well as its header are grouped  
in a tree-like structure.  They are considered level-1 packets and written to a NoSQL database. 
NoSQL database is schemaless,  making it ideal for storing data with complex
structures but small sizes, like STIX level-1 packets.  
The use of the NoSQL database provides great convenience for checking the integrity of the data packet,
finding a specific data packet, merging of packets, and for the subsequent analysis of STIX data.  
In addition to level-1 packets, 
other metadata such as the raw filenames,  the version of SPICE kernel data, the version of the MIB,
a  summary of the packets are 
also written to the NoSQL database during the parsing of packets. 

\subsubsection{Creation of level-1 FITS files}
Flexible Image Transport System (FITS) is a portable file standard widely used in the astronomy 
community to store, transmit and manipulate scientific images, tables and associated data (\cite{fits}).  
Therefore, FITS format is adopted by STIX data center to store standard data products. 
After the parsing of each new raw telemetry file, housekeeping, quick look, and science packets  
are sequentially selected from the NoSQL database and merged after passing checks for data integrity and consistency. 
The merged data as well as the associated metadata and auxiliary data are written to FITS files, and 
the metadata is also written to a collection in NoSQL in the meantime, which allows for fast queries on the files.
At STIX data center, files created from the level-1 packets are considered as level-1A data products. 

Level-1A FITS files are created automatically by the pipeline, and available at the STIX data center within minutes 
after the reception of the data. They are used in some of the subsequent data processing pipelines.
As predicted ephemeris data might be used when creating the Level-1A files,
FITS files are recreated after a few days to weeks after all resources are validated.
The created files are official Level-1  at STIX data center. 
\subsection{Energy calibration}
%STIX performs energy calibration onboard by using an energy look-up table, 
%which has to be
\begin{figure}
 \centering
  \includegraphics[width=0.8\linewidth]{figures/cal-fit.pdf}
  \caption{An example of STIX in-flight calibration count spectrum.
  The most prominent peaks, from left to right, are photopeaks of 31 keV, 35 keV, and 81 keV
  photons. The first two peaks are fitted by the double-Gaussian function and the high energy peak by  
  the crystal-ball function. }
    \label{fig:cal-fit}
\end{figure}
STIX rebins detected energies to 32 boarder science energy channels onboard
with an energy look-up table (ELUT), which can be constructed using energy conversion
factors determined from calibration runs. 
STIX acquires a high-resolution calibration spectrum
for the onboard Ba$^{133}$ sources for each pixel about every 24 hours. 
The right panel of Fig.~\ref{fig:cal-fit} shows an example of the calibration spectrum of a pixel accumulated for 24 hours.  
The three most prominent peaks are produced by the photons of  31 keV, 35 keV and 81 keV emitted by the calibration sources. 
To determine the positions of the photopeaks, we fit the first two peaks with the double-Gaussian function,  and the third peak
with the crystal-ball function (\cite{crsystallball}),  which consists of a Gaussian core portion 
and a power-law low-end tail, below a certain threshold.
Then we perform a linear fit of the positions and the photon energies, 
which gives the gain, namely, the ADC to energy conversion factor, and the baseline. 
Another calibration spectrum fitting method often used by STIX team  
 is the ECC method (see \cite{ecc,ecc2})).  We found that the results from the two methods are consistent at 1$\sigma$.

The above steps are performed for each calibration spectrum automatically once the data is available at STIX data center. 
 The results are written to a dataset in the NoSQL database. 
 They are used to further correction of energy bins for science data.  
Once significant changes in the calibration factors are 
observed,  a new energy lookup table is created 
and then uploaded to STIX.


\subsection{Solar flare identification}

STIX identifies solar flares onboard based on the
 count rates of the BKG detector and other detectors.
The rough time ranges and locations of identified flares are included in the QL data.
However,  they only provide limited information on flares due to the constraints of telemetry 
bandwidth and onboard computing resources. Moreover, micro-flares are not identified on-board 
due to the relatively high trigger threshold.

\begin{figure}
  \centering
  \includegraphics[width=0.8\linewidth]{figures/flaredet.pdf}
  \caption{STIX 4 -- 10 keV QL light curve recorded from 2022-08-10T21:00:00Z to 2022-08-10T18:00:00Z and 
  identified flares.   The orange curve is the smoothed light curve using a moving average filter with a time window
  of 1 minute.  The identified peaks are marked with plus signs, and colored ranges show their time ranges.
  }
  \label{fig:flare-det}
\end{figure}
Using QL light curves, solar flares can be identified in greater detail on the ground. 
The ground flare identification procedure includes the following steps:
\begin{itemize}
  \item Light curve smoothing. The selected light curve is filtered using an unweighted
  moving average filter with a time window of 1 minute. This can smooth out statistical fluctuations and electric surge spikes;
  \item Envelope subtraction.  
   A flare may last hours and there may have short-duration pulses lying on the main pulse in the light curve.
   The main pulse is the “envelope” of those short-duration pulses.
  To identify those short-duration pulses and estimate their durations, 
  The envelope is estimated using the Statistic-sensitive Non-linear Iterative Peak-clipping algorithm \cite{sinp}, which is widely used in X-ray spectroscopy.
    Then it is subtracted from the smoothed light curve.
  \item Identification of flare peaks. We consider a flare is detected 
 if the peak count rate after envelope subtraction lies beyond two standard deviations 
 of the mean count rate during quiet Sun periods. 
  The flare start and stop times are given at the times crossing the threshold.  
  \item Merging of flare peaks. Two flares are considered as one flare if their peak times are less than 5 minutes,
\end{itemize}

As an example, Fig.~\ref{fig:flare-det}  shows  STIX QL  light curve  in the energy range 4 -- 10 keV, recorded from 
2022-08-10T10:00:00Z to 2022-08-10T18:00:00Z.  The orange curve is the smoothed light curve using a moving average filter with a time window
of 1 minute. 
The identified peaks are marked with plus signs, and colored ranges show their time ranges.

%For each identified solar flare,
%  start time, end time, peak time, and peak 
%counts are from the selected light curve and 
%an 8-digit identification number of the format {\it yymmddHHMM}, which represents the solar flare peak time. 
%For example, the identification number 2201010000  indicates that 
%the solar flare peak time is at 2022-01-01T00:00:00 UT. 
The above steps are repeated for QL light curves of the other four higher energies
in the same time frame, which provides information on the upper limit of the X-ray energy produced by the flare.
Time ranges, peak count rate and total counts as well as the corresponding ephemeris data of identified flares are 
written to the flare list dataset in the NoSQL database.

%\subsection{Estimation of background level}
%As mentioned earlier, a threshold value calculated using background data needs to be provided when
%performing flare identification.  
%The background, although stable within a few days, can change over long periods,
% a suitable threshold is important for the accuracy of the identification. 
% Therefore, QL light curves for quiet sun periods are selected by excluding flaring periods. 
% Then median values and variances are calculated from the selected light curves and 
% written to a dataset in the database. They are used as inputs for the next flare identification. 
% Those processing steps are performed automatically after flare identification.

\subsection{Solar flare standard analysis pipeline}
\subsubsection{Estimation of solar flare GOES class }
\begin{figure}
  \centering
  \includegraphics[width=0.8\linewidth]{figures/goes_stix_flux_paper.pdf}
  \caption{Scatter plot of GOES low channel peak flux with respect to the equivalent  peak count rate  at 1 AU
   in the 4 -- 10 keV range
  for 717 solar flares observed by both GOES and STIX duration of the commissioning phase. 
  The solid line is a linear fit to the log-log graph. 
From the fit, we obtained 
the GOES flux (in units of W/m$^2$) $f = 10^{0.622 -7.376 \log_{10} (X^{'})}$,
where $X^{'}$ is STIX peak count rate corrected for the distance variations between the Sun and Solar Orbiter. 
}
\label{fig:goes-stix}
\end{figure}
%It is straightforward to calculate the GOES
%class of a flare observed by STIX from GOES flux if it is observed by GOES. 
%However,  Most of the time, 
Solar Orbiter is far away from the Earth and looks at 
the Sun from different angles. Therefore, a considerable number of flares observed by STIX
 are not observed by GOES satellites(and vice versa). 
%As already discussed in Ref.~\cite{andrea2021}, 
In order to estimate the GOES classes of those flares, 
we selected 717 solar flares observed by both GOES and  STIX  in 2021. 
Fig.~\ref{fig:goes-stix} shows the scatter plot of the peak flux
measured by GOES satellites
 with respect to the STIX background-subtracted peak count rate in the energy range from 4 to 10 keV. 
It should be noted that the STIX count rate 
has been corrected for 
the different distance between Solar Orbiter and the Sun using $X^{'}=x r^2$, where $X$ is the count rate after background subtraction
 and $r$ is the distance between the Solar Orbiter and the Sun in units of AU. 
As can be seen in the figure, there is a clear correlation between
them.  The widespread at low fluxes can be explained by the difference in 
the energy response of the two instruments and the variation in flare temperatures. 
From a linear fit to the correlation in the log-log scale, 
we obtained the GOES flux in units of W/m$^2$ $\log_{10}(f) = 0.622 -7.376 \log_{10} (X^{'})$.
The above formula is currently used to estimate GOES classes of the flares
which are not directly observed by GOES satellites.  
Estimated GOES flare classes are stored in the flare list in the database. 


\subsubsection{Estimation of coarse flare locations using CFL data}
The CFL sub-collimator consists of a front grid with
a distinctive pattern that selectively illuminates pixels of a 
dedicated detector based on the source location, which allows
estimating flare centroid location by using only the counts of the 12 pixels 
of that detector. 
%Flare centroid location can constrain the flaring region when reconstructing flare images.
Flare location is estimated onboard by maximizing the correlation between observed CFL pixel counts 
with expected counts using a look-up table. 
However, due to the limited onboard computing power,
STIX only calculates the flare locations for intermediate flares 
and their accuracy is only about 2 arcmin. 
With the science data downloaded from STIX,
the coarse flare location can be reconstructed on the ground. 
This allows for more sophisticated algorithms, greater flexibility in selecting time and energy 
range to be integrated, and more careful background subtraction.

\begin{figure*}
  \centering
  \includegraphics[width=0.95\linewidth]{figures/cflMay07.pdf}
  \caption{
   Left: Areas of illuminated regions of 12 pixels calculated by combining
  the twelve pixels counts with averages from other detectors, and the best fit. 
  Pixels 0 to 3 are the top big pixels, Pixels 4 to 7 are the bottom pixels, and 8 to 11 are the small pixels as shown in the right panel.
   Middle: Best-fit flare centroid location (marked by x) and its 1$\sigma$, 2$\sigma$, and 3$\sigma$ confidence contours.
   The best-fit location is obtained at (550, 278) sec. 
    Right:  Shadow (the gray shaded regions) of CFL sub-collimator projected on the detector 
  for the best-fit flare location. }
  \label{fig:cfl}
\end{figure*}

The coarse location of an identified flare is estimated once the level-1 pixel data  is received.
The procedure consists of the following steps:
\begin{itemize}
  \item Select  pixel data based on the flare time range and energy range information stored in the flare list dataset;
  \item Subtract background: The most recent level-1 dataset for the quiet sun period is used for background subtraction;
  \item Calculation of the mean fluence except for the CFL and the background detectors; 
      Except for those two detectors, the ratio between the illuminated area and total area $r$ only depends on the 
      slit width and the pitch width. It doesn't vary with the position of the flare in the STIX field of view (FOV).  
      Therefore, the fluence of a detector can be given by $F=c/(r S_{\mathrm(d)})$,  where $c$ is the registered 
      count and $S_\textrm{det}$ the sum of the sensitive areas of twelve pixels. 
  \item Calculation of the areas of illuminated regions on CFL pixels $\vec{S}$ using $\vec{S} = \vec{C}/F$, where $C$  is an array of counts 
  recorded by 12 pixels. Their errors in the areas are also calculated; 
  \item  Coarse flare location is estimated by minimizing the weighted sum of squared deviations 
  (i.e. weighted chi-squares) between the calculated illuminated areas and expectations
  simulated for potential flare locations in a 400 $\times$ 400 grid, whose locations are separated by 10 arcsec.  
  \item Saving the calculated flare location to the flare list dataset. 
\end{itemize}
As an example, the left panel of Fig.~\ref{fig:cfl} shows the calculated and best-fit illuminated areas 
of the CFL pixels measure observed for STIX flare 2105071900 (GOES class M3.9);   the middle panel shows the best-fit 
flare centroid location, as well as its $1\sigma$, $2\sigma$ and $3\sigma$ contours. 
The simulated shadow of the CFL sub-collimator is shown in the right panel. 
%\subsubsection{Joint observation}

\subsubsection{Image reconstruction and spectral analysis}
\begin{figure*}
  \centering
  \includegraphics[width=0.95\linewidth]{figures/imaging_pipeline.pdf}
  \caption{ 
   STIX Quick-look light curves and reconstructed images of the solar flare observed at 2022-03-08T08:55:17Z, 
   created by the image reconstruction pipeline.  The period at the peak was selected and 
   Back-projection, CLEAN, EM, and VIS\_FWDFIT
    algorithms were used for reconstructing images. Image products are accessible
   at STIX data center.}
  \label{fig:imaging}
\end{figure*}
STIX detects thousands of solar flares each year. However, only
 a part of them is analyzed in detail by solar physicists. 
To facilitate the selection of flares of interest, a flare imaging reconstruction and the spectral fitting pipeline 
has been developed and integrated into the main data processing pipeline on the platform. 
For each flare, pixel counts recorded within a one-minute time frame around the flare peaks  
are selected.
Reconstruction of an image requires inputs such as background data in the same level, 
STIX pointing information, spacecraft orientation. 
They are prepared based on the knowledge of NoSQL. 
After background subtraction, transmission, and dead time corrections, the selected counts are used for 
the calculation of visibilities for two energy ranges 4 -- 10 keV (thermal emission) and 16 -- 28 (non-thermal)  
keV.
Then images are reconstructed with four different algorithms: Back-projection, CLEAN, 
MEM and VIS\_FWDFIT \cite{paolo2020,clean, mem}.
Reconstructed images are further corrected for STIX off-pointing and spacecraft rotations with the auxiliary data. 
As an example,  the first panel of Fig. ~\ref{fig:imaging} shows 
the light curves and time range selected for a flare that occurred at about 2022-03-08T08:55:17Z.
The rest of the panels show the reconstructed images. 
The final by the pipeline are written to files in FITS  and PNG formats, while
their metadata are written to the NoSQL database.  
%They are accessible from STIX data center website or via stixdcpy\footnote{stixdcpy}.
\subsubsection{Spectral analysis}
Energy spectra provide direct information on electron acceleration in solar flares. 
The x-ray spectral fitting package OSPEX in SSWIDL includes a wide range of commonly used 
functions for parametrizing the thermal
and non-thermal components, as well as an interface to perform
the fits \footnote{\url{https://hesperia.gsfc.nasa.gov/ssw/packages/spex/doc/}}.
It reads pixel count data from L1 FITS files, corrections for dead time, transmission 
and energy binning for the counts,  
and calls  OSPEX routines for spectral fitting. 
\begin{figure}[h]
  \centering
  \includegraphics[width=0.9\linewidth]{figures/ospex.pdf}
  \caption{ 
    An example of spectral fitting results. The imaging and spectroscopy pipeline performs
    spectral analysis for all flares in the flare list. Spectral fitting results are written to both FITS files and 
    the NoSQL database.
   }
  \label{fig:ospex}
\end{figure}
\section{Science data request strategy}
As mentioned earlier, STIX only down-links low-latency data 
automatically due to the data rate constraints.  
Pixel data with higher time and energy resolution contain much richer information. 
They persist in the onboard archive memory for about five to six months 
before they are overwritten.
They can be compressed and 
down-linked after receiving data request telecommands initiated by the STIX operation team. 
A data request telecommand needs to provide information on the
time range, minimal time binning, energy range, energy binning  and selected detectors (or pixels), based on 
the knowledge from low-latency quick-look light curves.
Requesting science data from STIX is a tedious task, as it must take
into account many factors, such as constraints on the  allocated telemetry data rate, 
the daily maximum number of requests that can be submitted,  
the onboard time binning, 
solar flare fluxes and the scientific value of the data. 
After two years of operation, the following data selection strategy has been adopted: 
\begin{itemize}
  \item 
  Level-1 pixel data 
  of the flaring period is requested for detected flares with a total number of signal counts  greater than 1000, 
  which is the minimum count to reconstruct an image.  
  To minimize the telemetry data size, 
  the energy range of the request is limited to the range with signal counts based on the 
  knowledge from light curves. 
  If the pixel-summed peak count rate is less than 125 counts/sec
  (equivalent to the count rate observed for a B3 flare at 1 au), 
  a single time-bin is requested. Otherwise, a finer time resolution of science data is requested.
  The actual time resolution is adjusted based on the amount of data allocated.
 \item Spectrograms (onboard level-4) require a relatively small amount of data, therefore
  the highest possible time resolution (\~ 0.5 seconds depending on the configuration)
   and energy resolution  (no rebinning of science energy bins) are requested for all periods that STIX is in NOMINAL mode. 
\item At least one level 1 scientific data background data is requested per day. 
Background period selection is done by excluding periods from the list of flares stored in the database. 
If a quiet solar period is not found, a relatively quiet solar period will be selected. 
To reduce the amount of telemetry data, 
background data request counts are time-integrated over the requested period, i.e., a single time-bin.

\item Aspect data request
\end{itemize}
A routine has been developed to create the data requests mentioned above automatically. 
A unique ID is assigned to each request.
Unique IDs have a format of {\em yymmddxxxx}, where 
the first eight digits indicate the two-digit year,  month, and day of the data start time, respectively, 
and the last four digits are a random number, which is unique for the day.  
Information of created data requests is stored in the NoSQL database. 
After being checked by STIX team, data requests are flagged as pending requests.
Then they are selected and compiled into instrument operation requests (IORs), which are used to create the final telecommands
to be executed by the instrument.
Apart from those automatically created data requests, 
data requests are also created by the instrument team for special requirements.
\section{STIX data platform user interfaces}


\subsection{Interactive web pages}
\begin{figure}[h]
  \centering
  \includegraphics[width=0.9\linewidth]{figures/interfaces.pdf}
  \caption{ 
    Data flow at STIX data center.
  }
  \label{fig:interfaces}
\end{figure}

\begin{figure*}[h]
  \centering
  \includegraphics[width=0.7\linewidth]{figures/data-browser.pdf}
  \caption{ 
    Interactive web-based STIX Quick-look data browser. 
    Apart from STIX Quick-looks, it can also display quick-looks of simultaneous measurements 
    performed by other instruments, such as GOES X-ray fluxes, SDO/AIA images and Solar Orbiter EUI. 
    The browser is available at the link \url{https://datacenter.stix.i4ds.net/view/ql/lightcurves}.}
  \label{fig:qlbrowser}
\end{figure*}


To facilitate access to the data, a series of web APIs have been developed.
The web APIs accept HTTP requests from the client side, 
then the requests are processed on the server-side
 and the results are returned to the client side.  
The web APIs provide access to all STIX data products 
and the metadata stored in the NoSQL database. 

Based on the web APIs, we have built dozens of different web applications to manage and visualize 
different data products. 
The reason that we chose web techniques for the data browsers is that
web applications provide many advantages, such as clear cross-platform usability, 
wide access with browsers, allowing rapid development and easy maintenance. 

As an example, 
Fig.~\ref{fig:qlbrowser} shows a screenshot of the interactive web-based STIX Quick-look data browser hosted 
at STIX data center website.  It allows users to browse any historical STIX Quick-look data.
On the browser, users can specify time ranges of Quick-look data to be loaded.
After the web API on the server receives a request, 
it reads level-1 packets from the NoSQL database measured in the given time frame, and sends the
compiled and serialized data back to the web browser. 
Then the data are parsed and used to create an interactive light curve plot on the client side by using JavaScript.
Thanks to the use of state-of-art web technologies, the interactive plot allows users  
to rebin integration times, correct light travel times, estimate GOES fluxes and export data to local files. 
Apart from STIX quick-looks, quick-looks of 
simultaneous measurements performed by other Sun observing instruments
can also be loaded on the same page, which greatly facilitates finding of the 
events of interest for joint analysis. 

Based on similar concepts, tens of web tools have been developed for 
browsing other STIX data products. 
The four most used by STIX data users are: 
\begin{itemize}
  \item  Science data browsing and interactive analysis tool:
  It provides users tools to search for science data, 
  download science data, 
  perform common data analysis tasks and visualize science data and
  data analysis results. 
  On the web app, users can also select data of interest  
  for common analysis tasks, such as background subtraction, energy and time rebinning, 
  and estimation of coarse flare locations, which are on the client side using JavaScript.
Moreover, Users can also submit imaging and spectroscopy requests to the cloud with the
 interactive analysis tools on the page without the need of installing software. 
 Interactive plots are created using the results. Users can also export data from the interactive plots or 
  download the science data in the FITS format to the local disk for further analysis. 
  The browser greatly reduces the barrier to explore STIX science data for newcomers and also 
  for experienced users. 

  \item  Preview images and spectroscopy product browser:
  The tool provides a web-based imaging and spectroscopy manager. With the tool, users can view the 
  reconstructed solar flare images and fitting results from 
  automated imaging/spectroscopy runs and those submitted by register users. 
  It also allows users to plot the time evolution of emission measures and temperatures, as well as animation of x-ray images 
  for the selected runs.  Users can also create IDL or python templates, allowing the same results to be reproduced on their local machine.


  \item  Auxiliary data viewer: 
  The auxiliary data viewer allows users to view any historical auxiliary data of Solar Orbiter. 
  After receiving the time range information from the client-side,
  the server-side uses the SPICE kernels and SPICE toolkit (or STIX aspect solutions when they are available) to calculate 
  auxiliary data commonly used in data analysis for user-selected time ranges, 
  such as spacecraft location, velocity, light time difference, STIX pointing, 
   STIX FOVs, and angles from different observers, and sends the results to the client side. 
   Then the results are charted on the web page using JavaScript. 

  \item  Housekeeping data browser: 
The housekeeping browser allows users to browse any historical HK data from STIX. After the housekeeping browser receives the user's request, it sends the request to the server. The server directly uses the NoSQL data package to scale and serialize the data and return it to the browser. The browser then uses the data to generate interactive graphics (such as sensors). 
temperature, voltage, working status, memory status, etc.).
which provides great convenience for load operation control and for understanding instrument status in data analysis.
\item STIX data access page:  
STIX adopts an open data policy. 
STIX data products are published on the data access page once they are generated. 
On the page, users can search for data products by providing the data type and the observation time range  
, or download data products to local disks for further analysis. 

\end{itemize}




\subsection{STIX data center interface:  stixdcpy}
{\it stixdcpy} is a python package that facilitates access and analysis of STIX data.
It provides APIs to query and download data from STIX data center 
 and a set of tools for visualizing data and performing common analysis tasks. 
 With stixdcpy, users can query and download the following,  almost all data products available at STIX data center. 

similar to the web tools, stixdcpy also provide common data analysis algorithm, such as 
live time correction, transmission correction, data clipping, and  merging, auxiliary data tools, 
creating quick-looks for products 
stixdcpy is still under  development 
The source code of stixdcpy is hosted at the github repo at  \url{https://github.com/i4Ds/stixdcpy}.

\section{Summary}
\label{sec:summary}
STIX is one of ten instruments onboard Solar Orbiter.
 It measures the spectrum and takes X-ray images of solar 
 flares in the energy range 4 -- 150 keV. Solar Orbiter was launched into space on February 10, 2020. 
 During nominal operations, STIX continuously generates telemetry data. 
 To process and archive data as well as to support the operation of 
 instruments and scientific activities using STIX data, 
 automated data processing pipelines and data platforms have been 
 developed for STIX at FHNW. 
 The pipelines generate telemetry at different levels and perform common scientific analyses. 
 The platform provides 
 all STIX data products of different levels and also provides users 
 with various web-based tools to search for,  browser STIX data products. 
 It also provides web-based tools to perform common analysis tasks with STIX data. 
  The data center is designed to work in a 
 fully automatic mode with minimal human intervention. The concept has proven successful 
 and has been running continuously for over two years.
The platform not only facilitates the operations of the instrument but also provides great support to STIX data users.



\bibliographystyle{aa}
\bibliography{citations}

\end{document}
